% arara: lualatex
\documentclass{article}
\usepackage[italic]{mathastext}
% 'isomath' sets upper case greek letters italic in accordance with
% the International Standard ISO 80000-2
\usepackage{isomath}
\renewcommand{\familydefault}{\sfdefault}
\usepackage{helvet}

\usepackage[T1]{fontenc}
%\usepackage[latin9]{inputenc}
\usepackage[utf8]{inputenc}
\usepackage{amstext}
\usepackage{graphicx}
\usepackage{amssymb}
\usepackage{multirow}
\usepackage{amsmath,amssymb}
\usepackage{verbatim}
\usepackage{braket}
%\usepackage{float}
%\pagestyle{empty}
\usepackage{mathtools}

\usepackage[T1]{fontenc}% http://ctan.org/pkg/fontenc
\usepackage[outline]{contour}% http://ctan.org/pkg/contour
\usepackage{xcolor}% http://ctan.org/pkg/xcolor
\usepackage{anyfontsize}

\makeatletter
\newcommand\thefontsize[1]{{#1 The current font size is: \f@size pt\par}}
\makeatother

\begin{document}
    Example: Measure $U_{21}$

    Choose $\ket{\psi_0}$\\
    Choose $\hat{O}$\\

    $\bra{00}$\\
    $\bra{01}$\\
    $\bra{10}$\\
    $\bra{11}$\\
    $q_{init}$\\
    $q_{measure}$\\

    \section{ Connection between many-body Ramsey spectroscopy and unitary tomography (2 qubit particle conserving evolution)}
    In the recent demonstration of quantum supremacy \cite{Arute2019} the two qubit gates were photon number conserving unitaries from a class of unitaries
    known as the Fermionic Simulation or FSim class\cite{Kivlichan2018}.
    The matrix elements for the two qubit unitaries were inferred by optimizing the gate parameters to the cross-entropy benchmarking fidelity.
    The initial guesses for this optimization process were provided by a technique we refer to as unitary tomography.
    This technique is now widely used for two qubit gate calibration in the Martinis group / Google AI Quantum lab.
    Here we highlight the connection between the many-body Ramsey technique discussed in the previous section and the simple, powerful unitary tomography technique.

    We wish to measure the matrix elements of a generic two qubit photon conserving unitary $U$.
    In order to probe $U$, we can apply the unitary to an initial state of our choice.  We can also perform tomographic rotations prior to measurement.
% TODO Why
    \begin{equation}
        \left< \hat{O} \right> = \braket{\psi_0 | U^{\dagger} \hat{O} U | \psi_0 }
        %\hat{O}
    \end{equation}
    where the experimenter has control over $\ket{\psi_0}$ and $\hat{O}$
    A generic two qubit unitary $U$ has a matrix representation as below using the basis $\ket{00}, \ket{01}, \ket{10}, \ket{11}$
    \begin{equation}
        U =
        \begin{bmatrix}
            U_{00} & U_{01} & U_{02} & U_{03} \\
            U_{10} & U_{11} & U_{12} & U_{13} \\
            U_{20} & U_{21} & U_{22} & U_{23} \\
            U_{30} & U_{31} & U_{32} & U_{33} \\
        \end{bmatrix}
    \end{equation}
    which can be simplified for the FSim class.

    \begin{equation}
        U=
        \begin{bmatrix}
            1 & 0 & 0 & 0 \\
            0 & U_{11} & U_{12} & 0 \\
            0 & U_{21} & U_{22} & 0 \\
            0 & 0 & 0 & U_{33} \\
        \end{bmatrix}
    \end{equation}
    We wish to ascertain the matrix elements of this unitary transformation.

    The procedure for doing this is exactly the same as for many-body Ramsey spectroscopy and is illustrated in Fig.~\ref{unitary tomography schematic}.
    It consists of preparing a qubit in the superposition state with a $\pi / 2$ pulse,
    acting on the two qubit system with the unitary of interest,
    and measuring the observable $\langle \sigma^x \rangle + i \langle \sigma^y \rangle$.
    In this case our choice of which qubits were initialized and measured determines which unitary matrix element is extracted.
    Also in contrast with many-body Ramsey spectroscopy, we do not need to measure a full time series, but rather obtain our answer by analyzing a single application of the Unitary.

    %\quickwidefig{0.8\columnwidth}{./PDF/unitary_tomography_thesis.png}{A schematic diagram of the unitary tomography procedure for measuring the matrix elements of U}{unitary tomography schematic}


    The mechanics of this method are best clarified by working a simple example.
    If we initialize the left qubit

    \begin{equation}
        \ket{\psi_0}=\frac{1}{\sqrt{2}}
        \begin{bmatrix}
            1 \\
            1 \\
            0 \\
            0 \\
        \end{bmatrix}
    \end{equation}

    then

    \begin{equation}
        U\ket{\psi_0} =
        \frac{1}{\sqrt{2}}
        \begin{bmatrix}
            U00 + U01 \\
            U10 + U11 \\
            U20 + U21 \\
            U30 + U31 \\
        \end{bmatrix}
        =
        \frac{1}{\sqrt{2}}
        \begin{bmatrix}
            1 \\
            U_{11} \\
            U_{21} \\
            0 \\
        \end{bmatrix}
    \end{equation}
    So that it is clear we are selecting elements from the first column of the single photon subspace.
    On the other hand, if we initialize the second qubit then we select the corresponding column of the unitary
    \begin{equation}
        \ket{\psi_0}=\frac{1}{\sqrt{2}}
        \begin{bmatrix}
            1 \\
            0 \\
            1 \\
            0 \\
        \end{bmatrix}
        \implies
        \ket{\psi_{f}}=
        \frac{1}{\sqrt{2}}
        \begin{bmatrix}
            1 \\
            U_{12} \\
            U_{22} \\
            0 \\
        \end{bmatrix}
    \end{equation}

    Isolating a particular row of the unitary is achieved through the choice of which qubit is used to measure $\left< \sigma^{x} + i\sigma^{y} \right>$.
    Explicitly, the measurement operator for a measurement on qubit 1 is :
    \begin{equation}
        \hat{O}=I \otimes (\sigma^{x} + i\sigma^{y}) =
        \begin{bmatrix}
            1 & 0\\
            0 & 1\\
        \end{bmatrix}
        \otimes
        \begin{bmatrix}
            0 & 1\\
            0 & 0\\
        \end{bmatrix}=
        \begin{bmatrix}
            0 & 1 & 0 & 0\\
            0 & 0 & 0 & 0\\
            0 & 0 & 0 & 1\\
            0 & 0 & 0 & 0\\
        \end{bmatrix}
    \end{equation}
    so that
    \begin{equation}
        \psi_{f}=
        \begin{bmatrix}
            1 \\
            U_{11} \\
            U_{21} \\
            0 \\
        \end{bmatrix}
        \implies
        \braket{\psi_{f}|\hat{O}|\psi_{f}} =
        \begin{bmatrix}
            1 & U_{11}^{*} & U_{21}^{*} & 0
        \end{bmatrix}
        \begin{bmatrix}
            0 & 1 & 0 & 0\\
            0 & 0 & 0 & 0\\
            0 & 0 & 0 & 1\\
            0 & 0 & 0 & 0\\
        \end{bmatrix}
        \begin{bmatrix}
            1 \\
            U_{11} \\
            U_{21} \\
            0 \\
        \end{bmatrix}
        = U_{11}
    \end{equation}
    For a measurement on qubit 2
    \begin{equation}
        \hat{O}=(\sigma^{x} + i\sigma^{y}) \otimes I=
        \begin{bmatrix}
            0 & 1\\
            0 & 0\\
        \end{bmatrix}
        \otimes
        \begin{bmatrix}
            1 & 0\\
            0 & 1\\
        \end{bmatrix}
        =
        \begin{bmatrix}
            0 & 0 & 1 & 0\\
            0 & 0 & 0 & 1\\
            0 & 0 & 0 & 0\\
            0 & 0 & 0 & 0\\
        \end{bmatrix}
    \end{equation}
    so that
    \begin{equation}
        \psi_{f}=
        \begin{bmatrix}
            1 \\
            U_{11} \\
            U_{21} \\
            0 \\
        \end{bmatrix}
        \implies
        \braket{\psi_{f}|\hat{O}|\psi_{f}} =
        \begin{bmatrix}
            1 & U_{11}^{*} & U_{21}^{*} & 0
        \end{bmatrix}
        \begin{bmatrix}
            0 & 0 & 1 & 0\\
            0 & 0 & 0 & 1\\
            0 & 0 & 0 & 0\\
            0 & 0 & 0 & 0\\
        \end{bmatrix}
        \begin{bmatrix}
            1 \\
            U_{11} \\
            U_{21} \\
            0 \\
        \end{bmatrix}
        = U_{21}
    \end{equation}

    We have succeeded in measuring a single matrix element of the unitary, by mapping it's magnitude and phase onto the single qubit magnitude and phase.
    This procedure generalizes trivially to the single photon manifold of multi-qubit systems.

\end{document}

