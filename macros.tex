%\newcommand{\bra}[1]{\langle #1 |}
%\newcommand{\ket}[1]{| #1 \rangle }
%\newcommand{\braket}[2]{\langle #1|#2\rangle}
%\newcommand{\bbraket}[3]{ \langle #1 | #2 | #3 \rangle }
%\newcommand{\norm}[1]{\| #1\|}

% noise cal bias t

%\renewcommand{\Re}{\textrm{Re}}
%\renewcommand{\Im}{\textrm{Im}}
% Figures. Example usage:
% \quickfig{\columnwidth}{my_image}{This is the caption}{fig:my_fig}






\DeclareRobustCommand{\quickfig}[4]{
\begin{figure}
    \begin{centering}
        \includegraphics[width=#1]{#2}
        \par\end{centering}
    \caption{#3}
    \label{#4}
\end{figure}
}

\DeclareRobustCommand{\quickwidefig}[4]{
\begin{figure*}[h]
    \begin{centering}
        \includegraphics[width=#1]{#2}
        \par\end{centering}
    \caption{#3}
    \label{#4}
\end{figure*}
\FloatBarrier
}

\DeclareRobustCommand{\qeqn}[1]{
\begin{equation}
    #1
\end{equation}
}

\DeclareRobustCommand{\qeqnarray}[1]{
\begin{eqnarray}
    #1
\end{eqnarray}
}

% From MBL work
\newcommand{\bc}[1]{{\color{RoyalBlue}#1}}
\newcommand{\bl}[1]{{\color{RoyalBlue}#1}}
\newcommand{\rd}[1]{{\color{BrickRed}#1}} % This must be properly cited.
\newcommand{\gn}[1]{{\color{OliveGreen}#1}}
\newcommand{\org}[1]{{\color{Orange}#1}} % TODO Items
%\newcommand{\bc}[1]{{\color{RoyalBlue} MK: #1}}
\newcommand{\bcch}[1]{{\color{OliveGreen} #1}}
\newcommand{\bcrd}[1]{{\color{BrickRed} BC: #1}}
\newcommand{\verify}[1]{{\color{BrickRed}#1}}
\newcommand{\nqninet}{N_{q_\text{9}} \left( t \right)}
%\renewcommand{\familydefault}{\sfdefault}
%\usepackage{helvet}

%\makeatletter
%\def\@maketitle{%
%\newpage
%\null
%\vskip 2em%
%\centering
%\let \footnote \thanks
%{\LARGE \@title \par}%
%\vskip 1.5em%
%{\large
%\lineskip .5em%
%\begin{tabular}[t]{c}%
    %\baselineskip=12pt
    %\@author
%\end{tabular}\par}%
%\vskip 1em%
%{\large \@date}%
%\par
%\vskip 1.5em}
%\makeatother


%\setcounter{topnumber}{2} %  maximum number of floats in the top area
%\setcounter{bottomnumber}{2} % maximum number of floats in the bottom area
%\setcounter{totalnumber}{4} % maximum number of floats on a text page
%\renewcommand{\topfraction}{0.99} % The maximum size of the top area
%\renewcommand{\bottomfraction}{0.99} %The maximum size of the bottom area
%\renewcommand{\textfraction}{0} %  The area that must not be occupied by floats
%\renewcommand{\floatpagefraction}{0.999} % the minimum part of the page that need to be occupied by floats to for a float page
%\setlength{\floatsep}{5pt plus 2pt minus 2pt}  % The separation between floats in the top and bottom areas
%\setlength{\textfloatsep}{5pt plus 2pt minus 2pt} % The separation between the top or bottom area and the text area
%\setlength{\intextsep}{5pt plus 2pt minus 2pt}  % For inline floats (placed by "here")

%\setstretch{1.0}
%\setlength{\columnwidth}{225 pt}
%\setlength{\columnsep}{20 pt}
%\small


\newcommand{\expect}[1]{\langle#1\rangle}
\newcommand{\half}{\frac{1}{2}}

