\begin{abstract}
Quantum computers hold great promise for enhanced information processing, with potential applications as diverse as factorization, optimization and the simulation of quantum dynamics.
The past decade has seen quantum processors scale from single quantum bits to devices with tens of qubits
that are able to out-perform advanced classical supercomputers for certain applications. \rd{cite quantum supremacy paper}
These algorithm demonstrations rely on advances throughout the technology stack, from material science, device design, and fabrication,
to highly engineered infrastructure such as low-noise electronics, well isolated cryostat, and sophistocated software
to advances in quantum control, metrology, and finally the conception and execution of an algorithm.
The research in this thesis relies on advances in all of the above mentioned areas.

\textbf{mm TiN abstract}
Superconducting coplanar waveguide (CPW) resonators are widely used structures in the fields of photon detection and quantum information processing.
In recent years, there has been a growing interest in titanium nitride (TiN) thin films because they have such properties as a widely tunable critical temperature (0 - 4.7 K), large surface inductance, and the ability to produce high intrinsic quality factor (Qi) resonators.
We have deposited nearly stoichiometric TiN films (Ti:N=1:1) on Si(100) substrates by reactive magnetron sputtering.
In our room temperature depositions, we varied the N2 flow rate (2-4.5 sccm)  and deposition pressures (2 - 9 mTorr) while holding the Ar flow rate fixed at 15 sccm.
By increasing the deposition pressure and adjusting the N2 flow rate to maintain stoichiometry, the film (compressive) stress was changed from $\sim 100$\, MPa to $>3000$\,MPa and the $Q_i$ of the CPW resonators made from these TiN films was increased from $\sim 10^4$ to $\sim 10^6$ for single photon excitations when measured at $\sim 100$\,mK.

In this talk, we discuss the detailed relationship between the microwave electrodynamic responses in these CPW resonators and these film properties.

\textbf{mm vortex abstract}
Two important dissipation sources in superconducting circuits operated at low power are surface loss from two level systems (TLS) and magnetic vortex loss.
By patterning the superconducting electrodes with an array of holes, it is possible to reduce or eliminate loss due to magnetic vortices.
However, since the highest levels of coherence in planar superconducting circuits have been achieved by improving the electrode-substrate interface, it is natural to expect that adding hole arrays to the electrodes may cause excess surface loss.
We present simulations predicting the excess loss magnitude to be $<10\%$ for typical ground plane hole arrays, but for extreme cases of hole size or placement the loss may be much greater.
We confirm the simulation result with measurements of high quality factor resonators ($Q_{i}$ $>$ $10^{6}$) with and without the hole patterns.

\textbf{mm mbl abstract}
Predicting the dynamics of quantum systems is a primary example of an intensive computational task that could be efficiently solved with a near term quantum computer.

To pursue this goal, we have fabricated a 9-qubit linear chain device made of superconducting circuits with nearest neighbor couplings.
Our device is unique because it features frequency tunable qubits and an adjustable inter-qubit coupling strength, making it well suited to the simulation of quantum systems.
We use the device to generate high-fidelity, multi-qubit, analog dynamics evolving under a Bose-Hubbard Hamiltonian.

In this talk, we discuss the calibration of this device for quantum simulation.
Once calibrated we use this device to probe the basic physics of thermalization.
Specifically, we use transport measurements to distinguish localized and diffusive behavior.
To complement this, we use echo techniques to characterize the propagation of entanglement these regimes.
This shows qualitative differences between how energy and entropy flow in the system.

\end{abstract}
