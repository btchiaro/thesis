\chapter{Introduction to quantum computing}
\section{what is a qubit}

Whereas one classical bit of information can be represented as either 0 or 1, the state of a single ideal qubit can be represented as a vector pointing to the surface of a sphere.
This sphere is referred to as the Bloch sphere, and provides excellent physical intuition for the behavior of a qubit.

% TODO Graphic of Bloch sphere
$\psi = \cos \left( \frac{\theta}{2} \right) \ket{0} + \sin \left( \frac{\theta}{2} \right)\ket{1} $

The points away from the poles have amplitude in both the $\ket{0}$ state and the $\ket{1}$ state and are said to be "in superposition".
The Bloch Sphere is in fact the graphic representation of the single qubit density matrix.
Points on the surface of the sphere correspond to pure states, and points within the sphere correspond to mixed states.

Already at the single qubit level it is evident that using quantum bits provides a means for higher density information encoding than is possible with classical bits.
The story is even more compelling when we combine multiple qubits together
If two qubits a and b are each in a superposition state on the equator of the Bloch sphere:
$\ket{\psi_a} = \frac{\ket{0}+\ket{1}}{\sqrt{2}}$ and $\ket{\psi_b} = \frac{\ket{0}+\ket{1}}{\sqrt{2}}$
then the state fo the full a,b system is
$\ket{\psi_{ab}} = \ket{\psi_a} \otimes \ket{\psi_b} = \frac{1}{4} \left( \ket{00} + \ket{01} + \ket{10} + \ket{11} \right)$.
Whereas a two bit classical processor can represent one of the states $\ket{00}$, $\ket{01}$, $\ket{10}$, or $\ket{11}$, the quantum processor is able to represent them all simultaineously.
In fact, the capacity of the quantum register is scaling exponentially in the number of qubits in contrast to the classical processor which is scaling linearly.
Roughly speaking, If you want to double the capacity of the classical processor you would need to double the number of classical bits.
In contrast, if you wish to double the capacity of the quantum processor you only need to add one more qubit.

Thus the advantages of the quantum computer over it's classical counterpart arise from the ability to store and process information in an intrinsically parallel way.

\section{applications}
While it is true that not every computational problem can take advantage of quantum resources, there are several cases where a quantum speedup has been proven to exist.

Shor's algorithm for factoring large numbers has been proven to give an exponential speedup

Grover's algorithm for search has only a polynomial speedup.

Other promsing areas of research rely on the efficiency of using the quantum computer to efficiently solve problems in quantum dynamics.
Optimization problems of such as the "Travelling salesman" are an example of this.  Since these problems can be mapped to finding the groundstate of an Ising spin lattice,
we can expect to realize a quantum speedup by simply producing our desired Ising lattice and allowing the system to decay to it's ground state and simply measuring this state.

In this thesis we take the most direct approach of all.
Directly computing quantum dynamics is a computationally hard problem.
In order to make a brute force solution to a quantum dynamics problem, you need to time evolve the wave function under a Hamiltonian.

% TODO write the schrodinger equation

The Exponentation of a matrix who's linear dimension is scaling exponentially in the number of qubits, is intractible for even a few tens of qubits.

% Table Hilbert space size for N 2-level, and 3-level systems ... Maybe fixed photon etc. (Classical memory required)
This is essentially what Feynman had in mind when he proposed using quantum resources to simulate quantum systems.
In 198x Feynman proposed using quantum resources to simulate quantum processes.



This is because, as discussed above, the Hilbert space of the system is growing exponentally in the number of qubits.
Classical methods rely on matrix exponentiation / exact diagonalization.
Additionally, there are several

\section{Implementations of analog}
In principle any controllable and well isolated quantum system can be used as a platform for quantum computation.
Indeed, early processors have been demonstrated in a variety of formats.

- Spins (NV)
- Trapped ions
- Lukin (Rydberg)
- Cold atoms for quantum simulation
- Superconducting qubits

\section{Challenges}
Although there is great promise in quantum computing there is one primary drawback.
The information in a quantum computer is unstable.
There errors arising from this instablity can be roughly categorized into two main categories:
- relaxation
- dephasing

% TODO Figure bloch sphere dephasing relaxation

In our effort to construct a quantum computer out of superconducting qubits a primary research direction is measuring and mitigating these error mechanisms.
This is done by a variety of means.
In this thesis we explore several aspects of this coherence engineering before demonstrating before providing an algorithm demonstration that where we
use a supreconducting qubit system to address a modern question in condensed matter physics: Entanglement growth in the Manybody-localized phase.
Material science (Chapter 3)
Device design and fabrication (Chapter 4)
Dephasing metrology (chapter 5)

\section{gates vs analog}

\chapter{Introduction to superconducting qubits}

\section{Resonator}
The linear resonator is a simple, critical element of many superconducting circuits.
- It is valuable as a diagnostic tool for coherence engineering.

It suffers from many of the same loss mechanisms as qubit devices, but is much simpler to fabricate.
It's simplicity means that the resonator can be iterated rapidly.
good tool to dissect problems, (fab complications not conflated with actual physics)

- Purcell filter

- Readout resonators
dispersive readout relies critically on the readout resonator.

- Bus resonators.
Some groups also use the linear resonator to mediate inter-qubit coupling.

\subsection{Basic Physics}
The CPW resonator is a simple electrical LC oscillator.
The C is given by the capacitance to ground per unit length.
TODO:  Equation from Gao or Rami
The L has two contributions
(1) geometric inductance.  is given by the inductance to ground per unit length.
(2) kinetic (surface) inductance
TODO:  Equation from Gao or Rami
TODO:  Equation from Gao or Rami
TODO:  Insert derivation of lambda/4 from frequency 1/sqrt(lc) ...
TODO:  Quantization in the charge / flux basis

\section{Josephson Junction}
The Josephson Junction provides us with a lossless, nonlinear, tunable inductance.

The Josephson Junction consists of two superconducting electrodes separated by a thin insulating barrier.
% Figure
In (year) Brian Josephson derived the following relations for (TODO: Cooper pair tunneling accross the junction)
% TODO insert Josephson relations

Since the L is L(I), the inductance is nonlinear.

The consequence of this nonlinearity is that when we make our harmonic oscillator the energy levels are no longer evenly spaced.
This has at least two important consequences.
1) For uncoupled qubits, we can address the 0-1 transition without exciting the 1-2 transition.
2) When performing a multi-qubit analog algorithm (many qubits at the same time).
The nonlinearity gives an interaction term in the Bose-Hubbard Hamiltonian.
This interaction term is essential in breaking integrability (making the dynamics computationally complex) and leads to the many-body effects that we observe in the last chapter.

\section{Transmon}

\section{gmon}
TODO derive Bose-Hubbard from circuit model.
% \subsection{more_basic_physics}
% - Rotating wave approximation
% - How do we drive a pi pulse
% - How do we do readout

\section{coherence}
The major shortcoming of the quantum bit is that they are inherently unstable.
This instability can be subdivided into relaxation (T1) and dephasing (T2).
\begin{itemize}
    \item Let's show a swap spectroscopy in here.
    \item Maybe also a Ramsey fringe.
\end{itemize}

\quickfig{2 in}{./PDF/LCO.pdf}{This is a simple LC oscillator (resonator)}{LCO}






