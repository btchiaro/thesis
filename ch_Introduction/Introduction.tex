\chapter{Introduction to quantum computing}
\section{what is a qubit}

Whereas one classical bit of information can be represented as either 0 or 1, the state of a single ideal qubit can be represented as a vector pointing to the surface of a sphere.
This sphere is refered to as the Bloch sphere, and provides excellent physical intuition for the behavior of a qubit.
The Bloch Sphere is in fact the graphic representation of the density matrix.

$\psi = \cos \left( \frac{\theta}{2} \right) \ket{0} + \sin \left( \frac{\theta}{2} \right)\ket{1} $

% # TODO Figure of bloch sphere.

\section{motivation}
\section{gates vs analog}
\section{Implementations of analog}
\subsection{cold atoms}

\chapter{Introduction to superconducting qubits}
\section{Resonator}
The linear resonator is a simple, critical element of many superconducting circuits.
- It is valuable as a diagnostic tool for coherence engineering.

It suffers from many of the same loss mechanisms as qubit devices, but is much simpler to fabricate.
It's simplicity means that the resonator can be iterated rapidly.
good tool to dissect problems, (fab complications not conflated with actual physics)

- Purcell filter

- Readout resonators
dispersive readout relies critically on the readout resonator.

- Bus resonators.
Some groups also use the linear resonator to mediate inter-qubit coupling.

\subsection{Basic Physics}
The CPW resonator is a simple electrical LC oscillator.
The C is given by the capacitance to ground per unit length.
TODO:  Equation from Gao or Rami
The L has two contributions
(1) geometric inductance.  is given by the inductance to ground per unit length.
(2) kinetic (surface) inductance
TODO:  Equation from Gao or Rami
TODO:  Equation from Gao or Rami
TODO:  Insert derivation of lambda/4 from frequency 1/sqrt(lc) ...
TODO:  Quantization in the charge / flux basis

\section{Josephson Junction}
The Josephson Junction provides us with a lossless, nonlinear, tunable inductance.

The Josephson Junction consists of two superconducting electrodes separated by a thin insulating barrier.
% Figure
In (year) Brian Josephson derived the following relations for (TODO: Cooper pair tunneling accross the junction)
% TODO insert Josephson relations

Since the L is L(I), the inductance is nonlinear.

The consequence of this nonlinearity is that when we make our harmonic oscillator the energy levels are no longer evenly spaced.
This has at least two important consequences.
1) For uncoupled qubits, we can address the 0-1 transition without exciting the 1-2 transition.
2) When performing a multi-qubit analog algorithm (many qubits at the same time).
The nonlinearity gives an interaction term in the Bose-Hubbard Hamiltonian.
This interaction term is essential in breaking integrability (making the dynamics computationally complex) and leads to the many-body effects that we observe in the last chapter.

\section{Transmon}

\section{gmon}
TODO derive Bose-Hubbard from circuit model.
% \subsection{more_basic_physics}
% - Rotating wave approximation
% - How do we drive a pi pulse
% - How do we do readout

\section{coherence}
The major shortcoming of the quantum bit is that they are inherently unstable.
This instability can be subdivided into relaxation (T1) and dephasing (T2).











