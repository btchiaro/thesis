\chapter{Superconducting circuits for quantum computation}
\label{ch:sc_circuits}
\inprogress{
In this section we give a practical introduction to superconducting circuits for quantum computing applications.
% introduce the fundamental elements of superconducting circuits and summarize the basic theory that determines their dynamics.
The derivations here follow closely those found in
refs~\cite{Pozar, GaoThesis, RamiThesis, MazinThesis, DanThesis, JimmyThesis, CharlesThesis, vanDuzer1999, Koch2007,
OliverQuantumEngineeringGuide, RoushanSummerSchool}.
%Cite Jimmy's thesis, Dan's thesis, Dan's writeup, van Duzer, John's writeup.  Charles thesis, Koch transmon, Will Oliver's engineering handbook, Pedrams course notes.
For completeness we summarize the essential results from these sources as necessary to describe the experiments in this thesis.
} %end inprogress

\section{Coplanar waveguide (CPW) resonator}
\subsection{The CPW resonator as an LC oscillator}
\quickfig{6 in}{./PDF/CPW_sw_schem.pdf}{A schematic diagram of a coplanar waveguide (CPW), shown in cross section}{CPW}
A coplanar waveguide (CPW) transmission line consists of a signal electrode that is isolated from a surrounding ground plane.
A cross sectional view of the basic CPW geometry is shown in cross section in fig.~\ref{CPW}.
This device supports a propagating transverse electromagnetic (TEM) mode  with phase velocity \cite{Pozar}
\qeqn{v_p = \frac{1}{\sqrt{C_l L_l}}}

The capacitance per unit length $C_l$ is due to the geometry of the electrodes and the relative dielectric constant of the substrate $\epsilon_r$.
$C_l$ can be calculated from Schwarz-Christoffel conformal mapping \cite{GaoThesis}, the result is:
\qeqn{
C_l = \left( \frac{1 + \epsilon_r}{2} \right) \epsilon_0 \frac{4 K(k)}{K(k^\prime)}
}
where $K$ is the elleptic integral of the first kind, $k = S/ \left( S + 2W \right)$, and $k^\prime = \sqrt{1 - k}$.

The total inductance per unit length $L_l$ consists of the geometric $L_g$ as well as kinetic inductance $L_k$,
which can be significant in superconducting devices.
\qeqn{ L_l = L_g + L_k}
%The inductance per unit length $L_l$ has a geometric contribution $L_g$ that can be similarly calculated.
$L_g$ can be calculated in similar fashion to $C_l$ \cite{GaoThesis}
\qeqn{L_g = \frac{\mu_0}{4} \frac{K \left( k^\prime \right)}{K \left( k \right)} }
The kinetic inductance, due to the inertia of superconducting Cooper pairs, is dependent both on the superconducting material and the geometry of the transmission line.
\qeqn{ L_k = g L_s }
The surface inductance of the superconductor $L_s$ depends on the normal metal resistivity and the thickness of the electrode. \cite{KherThesis}
\qeqn{L_s = \sqrt{ \frac{ \hslash \rho_n }{ \pi \Delta_0 t} } } % Equation 2.13
and g is a geometric factor that can be calculated analyticaly for the CPW geometry \cite{GaoThesis}
\qeqnarray{
g &=& \frac{1}{4 a K^2(k)(1-k^2)} \left( \pi \log \frac{4 \pi a}{t} - k \log \frac{1 + k}{1 - k} \right) + \\
  &=& \frac{1}{4 a K^2(k)(1-k^2)} \left( \pi \log \frac{4 \pi b}{t} - k \log \frac{1 + k}{1 - k} \right) \\
} % end qeqnarray
For high resistivity superconductors, such as titanium nitride considered in Chapter 3 $L_k$, can be significant and even dominate $L_l$.

We can construct a quarter wave resonator by terminating a segment of the transmission line in a short circuit to the ground plane.
The resonance condition for the quarter wave resonator can be found from the relation $v_p = f \lambda$.
\qeqn{
f_0 = \frac{1}{4 l }\frac{1}{\sqrt{L_l C_l}}
\label{f0_resonator}
}
A consequence of eq.~\ref{f0_resonator} is that high kinetic inductance materials can make resonators smaller in size for a constant frequency resonator.
%A drawback is of sensitivity of $L_k$ is that if the materials properties such as $\rho_2$ are not well controlled,

%From Pozar (eqn 6.28 we find that the input resistance of a segment of lossy transmission line that is terminated in a short circuit is)
It is apparent that a segment of lossy transmission should behave like a damped parallel RLC oscillator by inspecting the input impedance of such a line \cite{Pozar}
\qeqn{
Z_{in} = Z_0 \tanh \left( \alpha + i \beta \right) l = Z_0 \frac{1 - i \tanh \alpha l \cot \beta l}{\tanh \alpha l - i \cot \beta l}
}
where $\alpha$ and $\beta$ are the attenuation and propagation parameters of a TEM wave on the line.
For a transmisssion line with small loss and near resonance so that $\omega = \omega_0 + \Delta \omega$ this is approximated as
\qeqn{
Z_{in} = \frac{1}{1/R + 2 i \Delta \omega / 2 \omega_0}
}
Which has the same form as $Z_{in}$ for a parallel RLC oscillator.

\quickwidefig{\columnwidth}{./PDF/cpw_resonator.pdf}{
a) An optical micrograph of a CPW resonator.
b) The voltage antinode of the CPW resonator.
c) The voltage node of the CPW resonator.
d) A crosssectional SEM image of a CPW resonator.
e) The circuit representation of an LC oscillator capacitively coupled to a transmission line.
}{cpw_resonator}

\subsection{Coherence metrology with CPW resonators}

\subsubsection{Resonator qubit relationship}
%It's simplicity means that the resonator can be iterated rapidly.
%good tool to dissect problems, (fab complications not conflated with actual physics)

The superconducting qubit considered later in this chapter is also a weakly damped LC oscillator.
The damping is important because it causes relaxation events of the type sketched in fig.~\ref{schem relaxation dephasing}(a),
which in turn contribute to the qubit's logical error rate.
Thus, in order to achieve the best qubit performance the dissipation must be minimized.

Throughout the history of superconducting qubits the resonator has been used for coherence engineering.
This is because superconducting qubits and resonators are subject to many of the same dissipation mechanisms.
This is not surprising given the similarity of their construction.
Fig.~\ref{qubit_resonator}(a) shows an Xmon transmon superconducting qubit.
It can be thought of as two intersecting strips of CPW transmission line,
such as those that create the CPW resonator in Fig.~\ref{qubit_resonator}(c).

\quickwidefig{\columnwidth}{./PDF/resonator_qubit_proxy.pdf}{
a) an Xmon transmon qubit,
b) The capacitive end of a $\lambda/4$ resonator.
c) The full CPW resonator structure.
}{qubit_resonator}

\subsubsection{Quality factor}
The resonator figure of merit describing dissipation is the quality factor $Q$,
which characterizes the rate at which energy stored in the resonator is lost from it.  $Q$ is the number of oscillations
the circuit performs within one energy relaxation time constant, which can in turn be related to the qubit relaxation time $T_1$.
\qeqn{ Q = \frac{\omega E}{P} = \omega T_1}

The rate at which which energy is lost from the resonator depends both on its coupling to external circuitry as well as its internal dissipation so that
\qeqn{1/Q_{total} = 1/Q_{c} + 1/Q_{i}}
where $Q_c$ and $Q_i$ refer to the coupling and internal quality factors respectively.
Note that the quality factors sum in reciprocal since the energy loss rates are additive.

For a quarter wave resonator capacitively coupled to a transmission line, the internal quality factor $Q_i$ and $Q_c$ can be extracted
separately from transmission measurements.
This is done by parametrically fitting measurements of transmission S - parameter $S_{21}$ vs frequency.\cite{Megrant2012}
\qeqn{ \tilde{S}_{21}^{-1} = 1 + \frac{Q_i}{Q_c^*}\,\frac{1}{1 + i 2 Q_i \left( \frac{f - f_0}{f_0} \right) } }
In chapters 3 and 4 of this thesis we focus on minimizing the internal dissipation $\frac{1}{Q_i}$ to improve the relaxation times of our qubits.
Further discussion of the setup of such an experiment can be found in chapter 4.
%When using the resonator for coherence engineering, the primary performance metric is the internal quality factor $Q_i$.
%We characterize the dissipation of the resonator by its internal quality factor.
\draftcomment{
\subsubsection{Dielectric loss}

Under the common operating conditions for superconducting qubits, the quality factor is often limited by dielectric loss.

For in chapters 3 and 4, the role of the resonator is for coherence metrology.

The linear resonator is an LC harmonic oscillator.
The resonator is used in such diverse applications as the mediation of qubit-qubit coupling \rd{cite IBM, UCSB}, Purcell filtering \rd{cite Dan}, dispersive readout \rd{cite Dan and someone else}, and as a coherence metrology tool.
Although all the of qubit measurements in this thesis make use of dispersive measurement, in chapter~\ref{ch:TiN} and chapter~\ref{ch:vortex} we focus on the use of the CPW resonator as a coherence metrology tool.

\subsection{Implementation}
A common way to implement the resonator is to terminate a length of coplanar waveguide (CPW) transmission line with a short circuit to the surounding ground plane.
} % end draftcomment

\draftcomment{
\inprogress{
We begin by following Pozar

The quarter wave resonance condition is satisfied when
\rd{
\qeqn{ f \frac{\lambda}{4} = \nu}
}
Where f is the fundamental frequency of the resonator, $\lambda$ is the wavelength, and $\nu$ is the propagation velocity.

It has a propagation velocity on chip of
\qeqn{\nu = \frac{1}{\sqrt{LC}} } where the L and C are the inductance and capacitance per unit length.

The propagation velocity is determined by the capacitance and inductance per unit length $C_l$ and $L_l$.
The geometric contributions to $C_l$ and $L_l$ are calculated in \rd{Gao thesis}.
For superconducting resonators $L_l$ has an additional contribution due to the kinetic inductance.
The kinetic inductance fraction $\alpha = \frac{L_k}{L_g + L_k}$ depends on the surface inductance of the superconducting material as well as the thickness of the thin film.
$\alpha$ is a few percent for Aluminum resonators, but can dominate $L_l$ for superconducting materials with large normal metal resistivity, such as Titanium Nitride considered in Chapter~\ref{ch:TiN}.

Near resonance the $\lambda / 4$ resonator can be modeled as a parallel LCR oscillator \rd{cite Mazin, gao, Zmoidzinas, day, ...} where the resistance R describes dissipation in the circuit.
The dissipation is often reported in terms the quality factor of the resonator $Q = \frac{\omega E}{P}$ where E is the energy stored in the resonator and P is the power dissipated by it.
The rate at which which energy is lost from the resonator depends both on its coupling to external circuitry as well as its internal dissipation so that
$1/Q_{total} = 1/Q_{c} + 1/Q_{i}$ where $Q_c$ and $Q_i$ refer to the internal and coupling quality factors respectively.

The internal quality factor $Q_i$ quantifies the loss that is internal to the resonator.  It is determined by many of the same
loss mechanisms as determine $T_1$ in qubit circuits, making it a valuable diagnostic tool for coherence engineering.
It suffers from many of the same loss mechanisms as qubit devices, but is much simpler to fabricate.
These loss mechanisms include dielectric loss, magnetic vortex loss, and quasiparticle loss.
Thermal quasiparticles loss can be mitigated by operating the device at low temperatures relative to the superconducting gap.
Nonequilibrium quasiparticles, such as those that arise from irradiating the device with photons at energies greater than the gap,
can be suppressed by using light tight shielding for the operating environment as well as in line IR filters for the control wires.
} % end draftcomment

\draftcomment{
\org{Have a fig on the measurement circuit and Qi extraction.  Or just reference Tony and the supplement of the Flux trap paper.}
} % end draftcomment
} % end inprogress

\draftcomment{
\begin{itemize}
    \item Inductance and Cap per unit length of CPW
    \item This gives a wave velocity.
    \item The wave velocity and distance give a quarter wave resonant frequency.
    \item $Z_{in}$ for parallel resonant circuit Pozar 6.12
    \item $Z_{in}$ for terminated CPW pozar 6.28
    \item Tony APL characterize circuit for $Q_i$
    \item The C is given by the capacitance to ground per unit length. TODO:  Equation from Gao or Rami
    \item The L has two contributions
    \item (1) geometric inductance.  is given by the inductance to ground per unit length.
    \item (2) kinetic (surface) inductance
    \item TODO:  Insert derivation of lambda/4 from frequency 1/sqrt(lc) ...
\end{itemize}
}

%%%%%%%%%%%%%%%%%%%%%%
%%%%%%%%%%%%%%%%%%%%%%
%%%%%%%%%%%%%%%%%%%%%%
%%%%%%%%%%%%%%%%%%%%%%         Quantum
%%%%%%%%%%%%%%%%%%%%%%
%%%%%%%%%%%%%%%%%%%%%%
%%%%%%%%%%%%%%%%%%%%%%

\section{Quantum LC Oscillator}
\quickfig{1 in}{./PDF/LCO.pdf}{This is a simple LC oscillator (resonator), maybe add the wavefunction or potential as a subfigure.}{LCO}
\verify{At low temperatures and small energies} the LC oscillator behaves quantum mechanically.
As such it's dynamics in this regime are described by the Schr\"odinger equation
\begin{equation}
    i \hbar \frac{d}{dt} \ket{\psi(t)} = H \ket{\psi}
\end{equation}
Here we derive the Hamiltonian and show that the LC oscillator behaves as a harmonic oscillator.

The instantaneous power dissipated by a circuit element is
\qeqn{P=IV}
and therefore the energy stored within the element can be written as
\qeqn{ E = \int_{t^\prime = t_0}^{t^\prime=t} I(t^\prime) V(t^\prime) dt^\prime }
$t_0$ is normally taken to be $- \infty$, at which time $I(- \infty)=0$ and $V(- \infty)=0$.

The customary approach for deriving the Hamiltonian $H$ is to write down the Lagrangian for a generalized coordinate of the circuit and compute $H$ via the Lagendre transformation.
\verify{This approach ensures that we arrive at a Hamiltonian in terms of a generalized coordinate and it's conjugate momentum (i.e. described in terms of the normal modes.) .}
We choose the flux $\Phi$, indicated in Fig.~\ref{LCO}, as our generalized coordinate.  $\Phi$ is also known as the node flux or branch flux.  $\Phi$ is defined in terms of the voltage across the inductor.
\qeqn{\Phi(t) = \int_{-\infty}^{t} V(t^\prime) dt^\prime}

\noindent
The current I and voltage V at the node are simply related to $\Phi$
\qeqn{ I = \Phi / L}
\qeqn{ V = \dot{\Phi}}
We can also write the energy of the capacitor and inductor in terms of \Phi.
\qeqn{ E_{capacitor} =
\int_{t^\prime = t_0}^{t^\prime=t} I(t^\prime) V(t^\prime) dt^\prime =
\int_{t^\prime = t_0}^{t^\prime=t} C \frac{dV}{dt^\prime} V(t^\prime) dt^\prime =
\frac{1}{2} C V^2 = \frac{1}{2} C \dot{\Phi}^2 }
and
\qeqn{ E_{inductor} =
\int_{t^\prime = t_0}^{t^\prime=t} I(t^\prime) V(t^\prime) dt^\prime =
\int_{t^\prime = t_0}^{t^\prime=t} I(t^\prime) L \frac{dI}{dt^\prime} dt^\prime =
\frac{1}{2} L I^2 = \frac{1}{2 L} \Phi^2 }

\noindent
By associating the potential energy with the position dependent term and the kinetic energy with the momentum dependent term we can write the Lagrangian for the system
\qeqn{ \mathcal{L} = \mathcal{T} - \mathcal{U} = \frac{1}{2} C \dot{\Phi}^2 - \frac{\Phi^2}{2L}}
We calculate the momentum conjugate to $\Phi$ by differentiating the Lagrangian.
\qeqn{ \frac{\partial \mathcal{L}}{\partial \dot{\Phi}} = C \dot{\Phi} = C V = Q}
We recognize the conjugate momentum as the charge on the capacitor and hence denominate it Q.
The Hamiltonian $H$ is obtained from $\mathcal{L}$ as
\qeqn{
H = Q \dot{\Phi} - \mathcal{L} = C \dot{\Phi}^2 -  (\frac{1}{2} C \dot{\Phi}^2 - \frac{\Phi^2}{2L} ) =
\frac{1}{2} C \dot{\Phi^2} + \frac{\Phi^2}{2L}
\label{H_LC_Lagrangian}
}
or
\qeqn{H = \frac{1}{2} C V^2 + \frac{1}{2} L I^2 }
which is the sum of the electric energy stored in the capacitor and the magnetic energy stored in the inductor.

The quantum mechanical Hamiltonian for the electrical LC oscillator can be obtained by substituting the charge and flux operators $\hat{Q}$ and $\hat{\Phi}$
in place of classical coodinate $\Phi$ and momentum $Q$ in Eq.~\ref{H_LC_Lagrangian}.
\begin{equation}
    H = \frac{\hat{Q}^2}{2C} + \frac{\hat{\Phi}^2}{2L}
\end{equation}
We recognize this as the Hamiltonian of a harmonic oscillator in the charge and flux coordinates.
Thus the device inherits the properties of a harmonic oscillator.
% The mechanical "mass on a spring" harmonic oscillator Hamiltonian is written
% qeqn{H_{SHO} = \frac{
The operators representing the conjugate variables obey the commutation relation
\qeqn{ \left[ \hat{\Phi}, \hat{Q} \right] = i \hbar}
Additionally, we get the familiar ladder of energy levels.
\qeqn{E_n = \hbar \omega_0 \left( n + \frac{1}{2} \right) }

We can also express the annihiliation and creation operators in terms of the charge and flux operators:
\qeqn{ \hat{a} = \frac{1}{\sqrt{2 \hslash Z_0} } \left( \hat{\Phi} + i \hat{Q} / Z_0 \right)}
and
\qeqn{ \hat{a}^\dagger = \frac{1}{\sqrt{2 \hslash Z_0} } \left( \hat{\Phi} - i \hat{Q} / Z_0 \right)}
where we have introduced the characteristic impedance
\qeqn{Z_0 = \sqrt{\frac{L}{C}} }

The Hamiltonian is often expressed in terms of the annihilation and creation operators:
\qeqn{ H = \hbar \omega_0 \left( \hat{a}^\dagger \hat{a} + \frac{1}{2} \right) = \hbar \omega_0 \left( \hat{n} + \frac{1}{2} \right) }

\noindent
The use of the number operator $\hat{n} = \hat{a}^\dagger \hat{a}$ makes it clear that the eigenstates of the system are Fock states corresponding to the number
of photonic excitations in the oscillator.
The equal spacing of the energy levels makes the linear harmonic oscillator unusable as a qubit because we cannot address a target transition.
This challenge is resolved by introducing nonlinearity into the circuit by incorporating a Josephson inductance.  This is described in the following sections.

It is convenient to write $H$ in terms of the number of charges on the island $\hat{n}$ and \rd{the phase across the inductor $\hat{\phi}$}.

\qeqn{ \hat{n} = \frac{\hat{Q}}{2 e} }

\qeqn{ \hat{\phi} = \frac{2 \pi \hat{\Phi}}{\Phi_0} }
where $e$ is the electronic charge and we have introduced the magnetic flux quantum
\qeqn{ \Phi_0 = \frac{h}{2 e} }

\noindent
We define the charging energy as
\qeqn{ E_c = \frac{e^2}{2 C} }
and the inductive energy
\qeqn{E_L = \frac{ \Phi_0^2 }{4 \pi^2 L} }
In these coordinates
\qeqn{H = 4 E_c \hat{n}^2 + \frac{1}{2} E_l \hat{\phi}^2 \label{H_lc_n_phi}}

\draftcomment{
\rd{
For our dynamics we use time independent Hamiltonians.
So we can use:
\begin{equation}
    \ket{\psi(t)} = e^{-i H t / \hbar}
\end{equation}

The LC oscillator is the electric analog to the simple harmonic oscillator.
Whereas in the mass on a spring simple harmonic oscillator energy oscillates between kinetic and potential, in the LC oscillator energy oscilates between the electric and magnetic fields.

The energy on the capacitor is $E_c = 1/2 C V^2$, and the inductor has an energy of $E = 1/2 L I^2$
}
} % end draftcomment

\section{Josephson Junction}
\quickfig{6 in}{./PDF/JJ_image_190908_634p.pdf}{a) A Josephson Junction b) The circuit symbol for a Josephson Junction}{jj_image}
The Josephson Junction provides us with a lossless, nonlinear inductance.\cite{Josephson1962}
The Josephson Junction consists of two superconducting electrodes separated by a thin insulating barrier.
This is illustrated in Fig.~\ref{jj_image}.
\draftcomment{We follow a derivation due to Van Duzer, cite Josephson
% TODO Give derivation of Josephson effect.}
% Figure
In 1962 Brian Josephson derived the following relations for (TODO: Cooper pair tunneling accross the junction)
Nobel 1973
}% end draftcomment
The Josephson relations for the current through the junction $I_j$ and the voltage across the junction $V$ to the phase:
\qeqn{I_j = I_c \sin \delta }
and
\qeqn{V = \frac{\phi_0}{2\pi}\frac{d\delta}{dt} }
from these we can compute the Josephson inductance
\qeqn{ V = L\frac{dI}{dt} \implies L_j = \frac{\phi_0}{2 \pi I_c \cos \delta} }

Since the L is L(I), the inductance is nonlinear.

The parameter $I_c$ is known as the critical current, which is related to the normal resistance of the junction via the
Ambegaokar - Baratoff Relation \cite{Ambegaokar1963}
\qeqn{I_c R_n = \frac{\pi \Delta}{2 e} \text{tanh} \left( \frac{\Delta}{2 k_b T} \right)}

The junction is an inductive element that stores energy in the magnetic field.  We do work against the junction to increase its energy.
As the junction is dissipationless, work done against the junction is stored as energy within it.
We can compute this energy from the current, voltage Josephson relations above.
\qeqn{-\frac{dU}{dt} = I V = \left( I_c \sin \phi(t) \right) \left( \frac{\hbar}{2 e}\frac{d\phi}{dt} \right) }
\qeqn{- \int dU = I_c \frac{\hbar}{2 e} \int \sin \phi d\phi }
\qeqn{U = E_J \cos \phi, E_J = I_c \frac{ \hbar }{2e} \label{jj_energy}}

We can configure two Josephson junctions in parallel to form a DC squid.
This device behaves as a Josephson junction with a critical current that is tunable with magnetic flux.
The effective critical current of the DC squid is \cite{vanDuzer1999}
\qeqn{
I_{c}(\Phi_{ext}) = 2 I_c \abs{ \cos{ \frac{ \pi \Phi_{ext} }{\Phi_0} } }
}
The tunability of the SQUID inductance forms the basis for frequency tunable superconducting qubits
and coupling circuits that provide an adjustable interaction strength between the qubits.

\section{Transmon superconducting qubits}

The transmon is a Josephson junction with a large shunting capacitance which \verify{exponentially} surpresses charge noise at the cost of anharmonicity\cite{Koch2007}.
In 2011 a 3D version of the transmon was developed at Yale and demonstrated remarkable coherence properties.\cite{Paik2011}
This is due to the fact that the geometry of the device made it possible to store most of the most of the qubit mode energy in the lossless vacuum.
Challenges of this 3d implementation are that they are too large to scale easily, cannot be microfabricated,
and it is difficult to retain the coherence properties when coupling two or more of these devices together.
% TODO image of transmon qubit. Maybe image of 3d transmon qubit. Probably need to request permission
In 2013 planar variant of the transmon was introduced.\cite{Barends2013}
Future iterations saw the device incorporate tunable inter-qubit coupling (gmon).\cite{Chen2014}

The transmon is a slightly anharmonic oscillator.
The anharmonicity results from replacing the linear inductor of the LC oscillator with a Josephson junction,
which provides a nonlinear inductance as described in the previous section.

\subsection{Hamiltonian of the transmon}
We can obtain the transmon Hamiltonian by substituting the Josephson junction energy from Eq.~\ref{jj_energy} into the LC oscillator Hamiltonian
Eq.~\ref{H_lc_n_phi} in place of the linear inductor energy.

\qeqn{H = 4 E_c \hat{n}^2 - E_j \cos{\hat{\phi}} \label{H_transmon} }
\inprogress{
A key point here is that increasing the capacitance decreases the charging energy.
Thus by forming the oscillator with a large capacitor the device is less sensitive to fluctuations in the charge.
} % end inprogress

\inprogress{
In order to operate the oscillator as a qubit, it is essential to make the energy levels anharmonic so that rotations may be driven between specified levels on demand.
The means by which we do this is to replace the linear inductance of our oscillator with a Josephson junction, which behaves like a nonlinear inductor.

The consequence of this nonlinearity is that when we make our harmonic oscillator the energy levels are no longer evenly spaced.
This has at least two important consequences.
1) For uncoupled qubits, we can address the 0-1 transition without exciting the 1-2 transition.
2) When performing a multi-qubit analog algorithm (many qubits at the same time).
The nonlinearity gives an interaction term in the Bose-Hubbard Hamiltonian.
This interaction term is essential in breaking integrability (making the dynamics computationally complex) and leads to the many-body effects that we observe in the last chapter.
}% end inprogress

\subsection{Nonlinearity of the transmon}
As suggested by Koch, the transmon can be thought of as a charged mechanical rotor in magnetic and gravitational fields.
In the transmon limit where Ej/Ec >> 1, the gravitational energy dominates, keeping the reduced flux angle $\varphi$ small.
This justifies a perturbation expansion in $\varphi$ about $\varphi=0$.
With this expansion the potential energy is
\qeqn{U = - Ej + Ej \varphi^2 / 2 - Ej \varphi^4 / 24}
Th quartic term is responsible for the nonlinearity to leading order.

\rd{Koch makes a gauge transformation to eliminate ng after which the Hamiltonian is claimed to be
\qeqn{ H = \sqrt{8 EcEj}(b^\dagger b +1/2)a - Ej - Ec/12(b+b^\dagger)^4 }
}

Here the $b,b^\dagger$ are the standard harmonic oscillator operators.
From perturbation theory, the first order correction to the energies for level n coming from the quartic term is is
\qeqn{\delta E_n =\bra{n} \delta_H \ket{n} =  Ec / 12 \bra{n}|(b + b^\dagger)^4|\ket{n}}
\qeqn{ \bra{n} | b^2 {b^{\dagger}}^2 + b^\dagger b b^\dagger b + b^\dagger b b b^\dagger + b b^\dagger b^\dagger b + b b^\dagger b b^\dagger + {b^{\dagger}}^2 b^2 | \ket{n}}
\qeqn{ \delta E_n = Ec / 12 \left( 6n^2 + 6n + 3 \right) }

Note that the anharmonicity decreases as we decrease the charging energy by increasing the capacitance.
This design choice is to trade away nonlinearity to gain charge noise insensitivity.

\section{Coupled resonators}
The advantages of quantum computing stem from the exponential scaling of Hilbert space dimension as we combine quantum systems.
Useful quantum circuits, therefore, must provide a means for multiple qubits to interact.
For superconducting qubits the coupling mechanism may be either capacitive or inductive in nature.
In this section we show how these coupling mechanisms lead to a hopping term in the Hamiltonian.

%\FloatBarrier
\subsection{Capacitive coupling}

The circuit diagram for two capacitively coupled oscillators is shown in Fig.\,~\ref{lco_cap_coupled}
\quickfig{3 in}{./PDF/lco_cap_coupled.pdf}{Two capacitively coupled LC oscillators}{lco_cap_coupled}
Maintaining the convention of associating the node fluxes $\Phi$ with potential energy and the node charges $Q$ with kinetic energy,
we can immediately write down the Lagrangian for the capacitively coupled system.

\qeqn{ \mathcal{L} = \mathcal{T} - \mathcal{U} =
\left[
\frac{1}{2} C_1 \dot{\Phi}^2_1 +
\frac{1}{2} C_2 \dot{\Phi}^2_2 +
\frac{1}{2} C_c \left( \dot{\Phi_1} - \dot{\Phi_2} \right)^2
\right] -
\left[ \frac{1}{2 L_1} \Phi_1^2 + \frac{1}{2 L_2}\Phi_2^2
\right]
}
Which can be written as
\qeqn{
\mathcal{L} =
\frac{1}{2} \bm{\dot{\Phi}}^T \bm{C} \bm{\dot{\Phi}} -
\frac{1}{2} \bm{\Phi}^T \bm{L}^{-1} \bm{\Phi}
}
\noindent
Where
\qeqn{
\bm{\Phi} =
\begin{bmatrix}
    \Phi_1 \\
    \Phi_2
\end{bmatrix}
\text{,}\,
\bm{C} =
\begin{bmatrix}
    C_1 + C_c & -C_c \\
    -C_c      &  C_2 + C_c \\
\end{bmatrix}
\text{, and}\,
\bm{L} =
\begin{bmatrix}
    L_1 & 0 \\
    0             &  L_2 \\
\end{bmatrix}
}

We find the conjugate momenta by differentiating the Lagrangian:
\qeqn{ Q_1 = \frac{ \partial \mathcal{L} }{\partial \dot{\Phi_1} } = C_1 \dot{\Phi_1} + C_c(\dot{\Phi_1}-\dot{\Phi_2})}
\qeqn{ Q_2 = \frac{ \partial \mathcal{L} }{\partial \dot{\Phi_2} } = C_2 \dot{\Phi_2} + C_c(\dot{\Phi_2}-\dot{\Phi_1})}

We can solve for the $\dot{\Phi}$ in terms of the $Q_i$ by writing the above expressions as a matrix equation.

\qeqn{
\begin{bmatrix}
Q_1 \\
Q_2 \\
\end{bmatrix}
=
\begin{bmatrix}
 C_1 + C_c & -C_c \\
-C_c      &  C_2 + C_c \\
\end{bmatrix}
\begin{bmatrix}
\dot{\Phi_1}\\
\dot{\Phi_2}
\end{bmatrix}
}
or
\qeqn{ Q = C \dot{\Phi} }
We can invert this expression to get the $\dot{\Phi_1}$ and $\dot{\Phi_2}$ in terms of the
cannonical momenta of the coupled system.
\qeqn{
\begin{bmatrix}
    \dot{\Phi_1}\\
    \dot{\Phi_2}
\end{bmatrix}
=
\mathbf{C}^{-1} \mathbf{Q}
=
\frac{1}{ (C_1 + C_c)(C_2 + C_c) - C_c^2 }
\begin{bmatrix}
    C_2 + C_c & C_c \\
    C_c      &  C_1 + C_c \\
\end{bmatrix}
\begin{bmatrix}
    Q_1 \\
    Q_2 \\
\end{bmatrix}
=
\begin{bmatrix}
    C^{-1}_{11} & C^{-1}_{12} \\[2.0ex]
    C^{-1}_{21} & C^{-1}_{22} \\[2.0ex]
\end{bmatrix}
\\
\begin{bmatrix}
    Q_1 \\
    Q_2 \\
\end{bmatrix}
\label{Phidot_in_terms_of_Q}
}

\inprogress{
We can write the Lagrangian in terms of the $Q_i$.
Since:
\qeqn{Q = C \dot{\Phi} \implies \dot{\Phi} = C^{-1} Q }
and
\qeqn{\dot{\Phi}^T = \left( C^{-1} Q \right)^T \implies \dot{\Phi}^T = Q^T {C^{-1}}^T  \implies \dot{\Phi}^T = Q^T C^{-1} }
Now we can write the lagrangian in terms of the node fluxes and the conjugate charges.
\qeqn{
\mathcal{L} = \frac{1}{2} Q^T C^{-1} Q - \frac{1}{2} \Phi^T L^{-1} \Phi
\label{Lagrangian_Phi_Q}
}
} % end inprogress
which is explicitly written as
\qeqnarray{
\mathcal{L} &=&
\half
\begin{bmatrix}
    Q_1 & Q_2 \\
\end{bmatrix}
\begin{bmatrix}
    C^{-1}_{11} & C^{-1}_{12} \\[2.0ex]
    C^{-1}_{21} & C^{-1}_{22} \\[2.0ex]
\end{bmatrix}
\begin{bmatrix}
    Q_1 \\
    Q_2 \\
\end{bmatrix}
-
\half
\begin{bmatrix}
    \Phi_1 & \Phi_2 \\
\end{bmatrix}
\begin{bmatrix}
    \frac{1}{L_1} & 0 \\
    0 & \frac{1}{L_2} \\
\end{bmatrix}
\begin{bmatrix}
    \Phi_1 \\
    \Phi_2 \\
\end{bmatrix}
\\
&=& \half \left[ C^{-1}_{11} Q_1^2 + C^{-1}_{12} Q_1 Q_2 + C^{-1}_{21} Q_1 Q_2 + C^{-1}_{22} Q_2^2 \right] -
\left[ \frac{1}{2 L_1} \Phi_1^2 + \frac{1}{2 L_2}\Phi_2^2 \right]
\label{lagrangian_Phi_Q_explicit}
}

The Hamiltonian is
\qeqn{H = \sum_i \dot{x}_i \frac{\partial \mathcal{L}}{\partial \dot{x}_i} - \mathcal{L} =
\sum_i \dot{x}_i p_i - \mathcal{L} = \sum_i \dot{\Phi}_i Q_i - \mathcal{L}
\label{Hamiltonian_general_Phidot_Q}
}
We can compute the sum by \ref{Phidot_in_terms_of_Q} and \ref{Lagrangian_Phi_Q} into \ref{Hamiltonian_general_Phidot_Q}

\inprogress{
\qeqn{
\dot{\Phi_1} Q_1 + \dot{\Phi_2} Q_2 =
(C^{-1}_{11} Q_1  + C^{-1}_{12} Q_2) Q_1 +
(C^{-1}_{21} Q_1  + C^{-1}_{22} Q_2) Q_2
}
} % end inprogress
So that
\qeqn{
H =
\half \left[ C^{-1}_{11} Q_1^2 + 2*C^{-1}_{12} Q_1 Q_2 + C^{-1}_{22} Q_2^2 \right] +
\left[ \frac{1}{2 L_1} \Phi_1^2 + \frac{1}{2 L_2}\Phi_2^2 \right]
} % end qeqn
This Hamiltonian resembles a harmonic oscillator Hamiltonian with increased effective capacitances with the addition of a charge - charge interaction.
The third term $2*C^{-1}_{12} Q_1 Q_2$ is the charge - charge interaction that permits excitations to transfer (swap)
between qubits.
Using the Harmonic oscillator relations above $Q = -i Q0 (a - a^\dagger)$ so that
\qeqn{
Q_1 Q_2 = - Q0_1 Q0_2 (a_1 - a^\dagger_1)(a_2 - a^\dagger_2) =
- Q0_1 Q0_2 (a_1 a_2 + a^\dagger_1  a^\dagger_2 - a_1 a_2^\dagger - a_2 a_1^\dagger)
}
The non photon conserving terms eliminated by the rotating wave approximation since they oscillate rapidly, quickly averaging to zero.\verify{rwacitation}
\qeqn{Q_1 Q_2 \sim (a_1 a_2^\dagger + a_2 a_1^\dagger)}
Written in this way it is clear that the capacitive charge - charge coupling mediates a swapping interaction.

\draftcomment{
This is all probably redundant now.
\rd{
\qeqn{
H = \sum_i \dot{\Phi}_i Q_i - \mathcal{L} = \frac{2 e^2}{C_1} n_1^2 + \frac{2 e^2}{C_2} n_2^2 + \frac{2 e^2}{C_c} n_1 n_2 +
\frac{\Phi^2_1}{2 L_1} + \frac{\Phi^2_2}{2 L_2}
} % end qeqn
where $C_1^{\prime} = C_1 + \frac{C_2  C_c}{C_2 + C_c}$, $C_2^{\prime} = C_2 + \frac{C_1  C_c}{C_1 + C_c}$ and $C_c = $
} % end rd

\rd{
This Hamiltonian resembles a harmonic oscillator Hamiltonian with increased effective capacitances which can be thought of as
the capacitance to ground of each node.
The third term $\frac{2 e^2}{C_c} n_1 n_2$ is the charge - charge interaction that permits excitations to transfer (swap)
between qubits.

Using the Harmonic oscillator relations above $Q = -i Q0 (a - a^\dagger)$ so that
\qeqn{
Q_1 Q_2 = - Q0_1 Q0_2 (a_1 - a^\dagger_1)(a_2 - a^\dagger_2) =
- Q0_1 Q0_2 (a_1 a_2 + a^\dagger_1  a^\dagger_2 - a_1 a_2^\dagger - a_2 a_1^\dagger)
}
The non photon conserving terms eliminated by the rotating wave approximation since they oscillate rapidly, quickly averaging to zero.\cite{rwacitation}
In the qubit subspace $(a - a^\dagger) = \sigma^y$
} %end rd
} %end draftcomment

\subsection{Inductive coupling}
The gmon circuit considered in this work features fixed capacitive coupling as described above in combination with a tunable inductive coupling.
\quickfig{3 in}{./PDF/lco_ind_coupled.pdf}{Two inductively coupled LC oscillators}{lco_ind_coupled}
Fig.~\ref{lco_ind_coupled} shows the case of two inductively coupled oscillators.
In this case we expect to generate a flux - flux interaction term in the Hamiltonian.

By the definition of mutual inductance we have
\qeqn{\Phi_1 = L_1 I_1 + M I_2}
\qeqn{\Phi_2 = L_2 I_2 + M I_1}

Which can be written as a matrix equation
\qeqn{
\begin{bmatrix}
\Phi_1 \\
\Phi_2
\end{bmatrix}
=
\begin{bmatrix}
    L_1 & M \\
    M & L_2
\end{bmatrix}
\begin{bmatrix}
    I_1 \\
    I_2
\end{bmatrix}
}

Which implies
\qeqn{
I = L^{-1} \Phi = \frac{1}{L_1 L_2 - M^2}
\begin{bmatrix}
L_2 & -M \\
-M  & L_1
\end{bmatrix}
\begin{bmatrix}
    \Phi_{1} \\
    \Phi_{2}
\end{bmatrix}
=
\begin{bmatrix}
    L^{-1}_{11} & L^{-1}_{12} \\[2.0ex]
    L^{-1}_{21} & L^{-1}_{22} \\[2.0ex]
\end{bmatrix}
\begin{bmatrix}
    \Phi_{1} \\
    \Phi_{2}
\end{bmatrix}
}

The energy in the inductors is
\qeqn{E_{inductor} = \half \mathbf{I}^T \mathbf{L} \mathbf{I} = \half \Phi^T L^{-1} \Phi}
Which we write in terms of the flux variable.

The energy in the capacitors is the same as in the uncoupled case:

\qeqn{E_{capacitor} = \half{} C_1 \dot{\Phi_1}^2 + \half{} C_2 \dot{\Phi_2}^2}% = \half \frac{Q_1^2}{C_1} + \half \frac{Q_2^2}{C_2}}

The full Lagrangian is
\qeqn{
\mathcal{L} =
\left[\half{} C_1 \dot{\Phi_1}^2 + \half{} C_2 \dot{\Phi_2}^2 \right] -
\left[L^{-1}_{11} \Phi^2_1 + L^{-1}_{12} \Phi_1 \Phi_2 + L^{-1}_{21} \Phi_1 \Phi_2 + L^{-1}_{22} \Phi_2^2 \right]
}

The conjugate momenta are
\qeqn{ Q_1 = \frac{ \partial \mathcal{L} }{\partial \dot{\Phi_1} } = C_1 \dot{\Phi_1} }
\qeqn{ Q_2 = \frac{ \partial \mathcal{L} }{\partial \dot{\Phi_2} } = C_2 \dot{\Phi_2} }

And finally the Hamiltonian is
\qeqn{ H = \sum_i \dot{\Phi}_i Q_i - \mathcal{L} =
\frac{Q_1^2}{2 C_1} + \frac{Q_2^2}{2 C_2} +
L^{-1}_{11} \Phi^2_1 + 2 * L^{-1}_{12} \Phi_1 \Phi_2 + L^{-1}_{22} \Phi_2^2
}

This is again the Hamiltonian for a harmonic oscillator, but this time with a flux - flux coupling term $2 * L^{-1}_{12} \Phi_1 \Phi_2$
Since \verify{$\Phi = 2 \sqrt{2 \hslash Z_0}(a^{\dagger} + a)$} and involking the RWA again, we find
\qeqn{\Phi_1 \Phi_2 \sim (a_1 a_2^\dagger + a_2 a_1^\dagger)}
so that the flux - flux coupling also generates a hopping term in the Hamiltonian.

\section{Composite Hamiltonian.}
\quickfig{6 in}{./PDF/fs1_190916_1029a.pdf}{
a) An optical micrograph of a gmon device.
b) Circuit schematic for 3 qubit subsection of the gmon.
c) The resulting Bose-Hubbard Hamiltonian that describes the gmon dynamics.
}{gmon_schematic}

More complex devices, such as the gmon circuit employed in Chapter~\ref{ch:MBL}, can be understood by combining the concepts in the preceding sections.


\draftcomment{
\org{\subsection{XY control}
Show how capacitive coupling results in the ability to do single qubit rotations.}
\gn{In order to use the device as a qubit we must be able to perform rotations between the levels of the device.
The standard way to do this is by coupling them together via the charge operator.
TODO:  Insert figure with the circuit diagram showing a SEM with the uwave drive line labeled.}
\org{\subsection{Sigma Z}
Show how detuning generates sigma z, this should be easy since the charge and flux operators combine to give n
The harmonic oscillator Hamiltonian $\hbar \omega \left( a^\dagger a + 1/2 \right)$
generates phase accumulation of the $\ket{1}$ with respect to the $\ket{0}$ state.
Using our bloch sphere convention, where $\ket{0}$ is at the North pole $\sigma{z} = 1 - 2 a^\dagger a$.
}

\org{\subsection{Dispersive shift from readout resonator}
How does coupling to the readout resonator give a dispersive shift?  Why sigma z?}
\org{\subsection{Dispersive readout}
Why does the readout resonator frequency shift with photon number?}
\rd{Now the more complicated gmon physics}
\org{\subsection{How does changing the copuler bias change sigma z?}
Why does the readout resonator frequency shift with photon number?}
\org{\subsection{How does capacitive coupling give hopping}
Why does the readout resonator frequency shift with photon number?}
\org{\subsection{How does "mode" coupling give a hopping term?}
What is mode coupling?}
\org{\subsection{Inductive coupling}
How does mutual inductance coupling give a hopping term?}
\org{\subsection{Inductive tail}
How does the inductive tail change the qubit frequency, anharmonicity, mode structure?}

\section{coherence}
The major shortcoming of the quantum bit is that they are inherently unstable.
This instability can be subdivided into relaxation (T1) and dephasing (T2).
\begin{itemize}
    \item Let's show a swap spectroscopy in here.
    \item Maybe also a Ramsey fringe.
\end{itemize}

\section{additional topics}
this section will probably just be deleted but perhaps I can comment on some of these areas if there is time.

\subsection{gmon}
The gmon is a variant of the transmon that features tunable inter-qubit coupling.
In order to implement this, rather than directly grounding the junctions we place a long inductive wire in series with them.

TODO derive Bose-Hubbard from circuit model.
% \subsection{more_basic_physics}
% - Rotating wave approximation
% - How do we drive a pi pulse
% - How do we do readout
\subsubsection{Modelling the gmon}
asdf
%The derivations in this section closely follow those found in \verfiy{Cite Charles paper, charles thesis}.

\subsection{gates vs analog}

\begin{itemize}
    \item Analog is challenging because requires so much clean frequency space.
\end{itemize}

}% end draftcomment
