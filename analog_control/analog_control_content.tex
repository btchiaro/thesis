% This is copied out of:
% devBenGreen/latex/xiy_notes.tex

\section{Unitary Tomography (2 qubit particle conserving evolution)}
We wish to measure the matrix elements of a generic two qubit photon conserving unitary U.

In order to probe U, we can apply the unitary to an initial state of our choice.  We can also perform tomographic rotations prior to measurement.
% TODO Why
\begin{equation}
    \left< \hat{O} \right> = \braket{\psi_0 | U^{\dagger} \hat{O} U | \psi_0 }
    %\hat{O}
\end{equation}

The unitary U has a matrix representation as below
\begin{equation}
    U =
    \begin{bmatrix}
        U_{00} & U_{01} & U_{02} & U_{03} \\
        U_{10} & U_{11} & U_{12} & U_{13} \\
        U_{20} & U_{21} & U_{22} & U_{23} \\
        U_{30} & U_{31} & U_{32} & U_{33} \\
    \end{bmatrix}
\end{equation}

The unitary for the chemistry gates (this is I believe a generic photon number conserving unitary) has a very specific form that we can take advantage of:

\begin{equation}
    U=
    \begin{bmatrix}
        1 & 0 & 0 & 0 \\
        0 & e^{ i (\phi{1} + \phi{2} + \phi{3})}\cos{\theta} & e^{ i (\phi{1} - \phi{2} + \phi{3})}\sin{\theta} & 0 \\
        0 & e^{ i (\phi{1} - \phi{2} - \phi{3})}\sin{\theta} & e^{ i (\phi{1} + \phi{2} - \phi{3})}\cos{\theta} & 0 \\
        0 & 0 & 0 & e^{i (\phi_{3}-\phi_{0})} \\
    \end{bmatrix}
\end{equation}=
\begin{equation}
    U=
    \begin{bmatrix}
        1 & 0 & 0 & 0 \\
        0 & U_{11} & U_{12} & 0 \\
        0 & U_{21} & U_{22} & 0 \\
        0 & 0 & 0 & U_{33} \\
    \end{bmatrix}
\end{equation}


The single photon subspace of U looks like
\begin{equation}
    U=
    \begin{bmatrix}
        e^{ i (\phi{1} + \phi{2} + \phi{3})}\cos{\theta} & e^{ i (\phi{1} - \phi{2} + \phi{3})}\sin{\theta} \\
        e^{ i (\phi{1} - \phi{2} - \phi{3})}\sin{\theta} & e^{ i (\phi{1} + \phi{2} - \phi{3})}\cos{\theta} \\
    \end{bmatrix}
\end{equation}


if we initialize the left qubit

\begin{equation}
    \ket{\psi_0}=\frac{1}{\sqrt{2}}
    \begin{bmatrix}
        1 \\
        1 \\
        0 \\
        0 \\
    \end{bmatrix}
\end{equation}

then

\begin{equation}
    U\ket{\psi_0} =
    \frac{1}{\sqrt{2}}
    \begin{bmatrix}
        U00 + U01 \\
        U10 + U11 \\
        U20 + U21 \\
        U30 + U31 \\
    \end{bmatrix}
    =
    \frac{1}{\sqrt{2}}
    \begin{bmatrix}
        1 \\
        U_{11} \\
        U_{21} \\
        0 \\
    \end{bmatrix}
\end{equation}

If we initialize the second qubit then we select the corresponding column of the unitary

\begin{equation}
    \ket{\psi_0}=\frac{1}{\sqrt{2}}
    \begin{bmatrix}
        1 \\
        0 \\
        1 \\
        0 \\
    \end{bmatrix}
    \implies
    \ket{\psi_{f}}=
    \frac{1}{\sqrt{2}}
    \begin{bmatrix}
        1 \\
        U_{12} \\
        U_{22} \\
        0 \\
    \end{bmatrix}
\end{equation}

Measuring the amplitude of $U_{ij}$ follows directly from z basis measurements.
Measuring the phase can be done by measuring $\left< \sigma^{x} + i\sigma^{y} \right>$
The phase of the unitary matrix element can be measured by directly measuring the single qubit phase.

Explicitly the measurement operator for a measurement on qubit 1 is :

\begin{equation}
    \hat{O}=I \otimes (\sigma^{x} + i\sigma^{y}) =
    \begin{bmatrix}
        1 & 0\\
        0 & 1\\
    \end{bmatrix}
    \otimes
    \begin{bmatrix}
        0 & 1\\
        0 & 0\\
    \end{bmatrix}=
    \begin{bmatrix}
        0 & 1 & 0 & 0\\
        0 & 0 & 0 & 0\\
        0 & 0 & 0 & 1\\
        0 & 0 & 0 & 0\\
    \end{bmatrix}
\end{equation}
so that
\begin{equation}
    \psi_{f}=
    \begin{bmatrix}
        1 \\
        U_{11} \\
        U_{21} \\
        0 \\
    \end{bmatrix}
    \implies
    \braket{\psi_{f}|\hat{O}|\psi_{f}} = U11
\end{equation}
or if you prefer:
\begin{equation}
    \braket{\psi_{f}|\hat{O}|\psi_{f}} =
    \begin{bmatrix}
        1 & U_{11}^{*} & U_{21}^{*} & 0
    \end{bmatrix}
    \begin{bmatrix}
        0 & 1 & 0 & 0\\
        0 & 0 & 0 & 0\\
        0 & 0 & 0 & 1\\
        0 & 0 & 0 & 0\\
    \end{bmatrix}
    \begin{bmatrix}
        1 \\
        U_{11} \\
        U_{21} \\
        0 \\
    \end{bmatrix}
    = U_{11}
\end{equation}


and for a measurement on qubit 2
\begin{equation}
    \hat{O}=(\sigma^{x} + i\sigma^{y}) \otimes I=
    \begin{bmatrix}
        0 & 1\\
        0 & 0\\
    \end{bmatrix}
    \otimes
    \begin{bmatrix}
        1 & 0\\
        0 & 1\\
    \end{bmatrix}
    =
    \begin{bmatrix}
        0 & 0 & 1 & 0\\
        0 & 0 & 0 & 1\\
        0 & 0 & 0 & 0\\
        0 & 0 & 0 & 0\\
    \end{bmatrix}
\end{equation}
so that
\begin{equation}
    \psi_{f}=
    \begin{bmatrix}
        1 \\
        U_{11} \\
        U_{21} \\
        0 \\
    \end{bmatrix}
    \implies
    \braket{\psi_{f}|\hat{O}|\psi_{f}} = U21
\end{equation}
or if you prefer:
\begin{equation}
    \braket{\psi_{f}|\hat{O}|\psi_{f}} =
    \begin{bmatrix}
        1 & U_{11}^{*} & U_{21}^{*} & 0
    \end{bmatrix}
    \begin{bmatrix}
        0 & 0 & 1 & 0\\
        0 & 0 & 0 & 1\\
        0 & 0 & 0 & 0\\
        0 & 0 & 0 & 0\\
    \end{bmatrix}
    \begin{bmatrix}
        1 \\
        U_{11} \\
        U_{21} \\
        0 \\
    \end{bmatrix}
    = U_{21}
\end{equation}



We have succeeded in measuring a single matrix element of the unitary, by mapping it's magnitude and phase onto the single qubit magnitude and phase.



\section{Eigenvalue case}

The wavefunction in the energy basis is

\begin{equation}
    \ket{\psi (t)} =e^{-i \mathcal{H} t} \ket{\psi_{0}} = \sum_{\alpha} C_{\alpha} e^{- i E_{\alpha} t} \ket{\phi_{\alpha}}
\end{equation}

The coefficients are given by the overlap with the initial state

\begin{equation}
    C_{\alpha} = \braket{\phi_{\alpha}|\psi_{0}}
\end{equation}

An observable $\hat{O}$ in the energy basis is written as:


\begin{equation}
    \hat{O} = \sum_{\alpha,\alpha^{\prime}} O_{\alpha,\alpha^{\prime}} \ket{\phi_{\alpha^{\prime}}} \bra{\phi_{\alpha}}
\end{equation}

The time dependent expectation value of this operator is

\begin{equation*}
    \begin{aligned}
        O(t) =\braket{\psi(t)|\hat{O}|\psi(t)} = \\[2.0ex]
        \left( \sum_{\beta} C^{*}_{\beta} e^{i E_{\beta} t} \ket{\phi_{\beta}} \right)
        \left( \sum_{\alpha,\alpha^{\prime}} O_{\alpha,\alpha^{\prime}} \ket{\phi_{\alpha^{\prime}}} \bra{\phi_{\alpha}} \right)
        \left( \sum_{\beta^{\prime}} C_{\beta^{\prime}} e^{- i E_{\beta^{\prime}} t} \ket{\phi_{\beta^{\prime}}} \right) = \\[2.0ex]
        \sum_{\alpha, \alpha^{\prime}, \beta, \beta^{\prime} }
        \braket{\phi_{\beta^{\prime}}|\phi_{\alpha^{\prime}}}
        \braket{\phi_{\beta}|\phi_{\alpha}}
        C_{\beta^{\prime}}^{*} C_{\beta} O_{\alpha, \alpha^{\prime}} e^{i (E_{\beta^\prime} - E_{\beta}) t} = \\[2.0ex]
        \sum_{\alpha, \alpha^{\prime}} C^{*}_{\alpha^{\prime}} C_{\alpha} O_{\alpha, \alpha^{\prime}} e^{- i (E_{\alpha^{\prime}} - E_{\alpha}) t}
    \end{aligned}
\end{equation*}

Consider the case where the operator O is defined as X+iY on a single qubit

\begin{equation}
    O = \sigma^{x} + i \sigma^{y} =
    \begin{bmatrix}
        0 & 1 \\
        1 & 0
    \end{bmatrix}
    +
    i
    \begin{bmatrix}
        0 & -i \\
        i & 0
    \end{bmatrix}
    =
    \begin{bmatrix}
        0 & 1 \\
        0 & 0
    \end{bmatrix}
\end{equation}

In this case the only term that we keep is
\begin{equation}
    \ket{\phi_{\alpha}^{\prime}}
    \bra{\phi_{\alpha}}=
    \ket{0}
    \bra{1}
\end{equation}

so that

\begin{equation}
    \left< O \right>(t) = C^{*}_{0}C_{1} \: O_{1,0} \: e^{- i(E_{0} - E_{1} )t}
\end{equation}

There are a few comments here.

It is intuitive that we need to begin in a superposition state to form an interference pattern.
We can also see that this is the case from the algebra above:  If either $C_0$ or $C_1$ = 0 then $\left< O \right>(t)$=0

We also see that $\frac{ \ket{\text{vac}} + \ket{\text{target}} }{\sqrt{2}}$ Will give the strongest signal.
If other states are part of the superposition, they will reduce the visibility due to normalization, but they will not add more frequencies.

Let's be more explicit about how this works for a 3x3 example.
\begin{equation*}
    \begin{aligned}
        O(t) =\braket{\psi(t)|\hat{O}|\psi(t)} = \sum_{\alpha, \alpha^{\prime}} C^{*}_{\alpha^{\prime}} C_{\alpha} O_{\alpha, \alpha^{\prime}} e^{- i (E_{\alpha^{\prime}} - E_{\alpha}) t}   \\[2.0ex]
    \end{aligned}
\end{equation*}

In the energy basis $\ket{\phi}$:
\begin{equation*}
    \ket{\psi_0} = a\ket{\phi_1} + b\ket{\phi_2} + c\ket{\phi_3}
\end{equation*}
so that
\begin{equation*}
    \ket{\psi (t)} = a\ket{\phi_1}e^{-i E_1 t} + b\ket{\phi_2}e^{-i E_2 t} + c\ket{\phi_3}e^{-i E_3 t}
\end{equation*}
and with
\begin{equation*}
    O =
    \begin{bmatrix}
        O_{11} & O_{12} & O_{13}\\
        O_{21} & O_{22} & O_{23}\\
        O_{31} & O_{32} & O_{33}\\
    \end{bmatrix}
\end{equation*}
we have
\begin{equation*}
    \begin{aligned}
        \braket{\psi(t)|\hat{O}|\psi(t)} = \\[2.0ex]
        \begin{bmatrix}
            c_1^* e^{i E_1 t} & c_2^* e^{i E_2 t} & c_3^* e^{i E_3 t}
        \end{bmatrix}
        \begin{bmatrix}
            O_{11} & O_{12} & O_{13}\\
            O_{21} & O_{22} & O_{23}\\
            O_{31} & O_{32} & O_{33}\\
        \end{bmatrix}
        \begin{bmatrix}
            c_1 e^{-i E_1 t} \\
            c_2 e^{-i E_2 t} \\
            c_3 e^{-i E_3 t}
        \end{bmatrix} = \\[2.0 ex]
        c_1^* e^{i E_1 t} \left( c_1 O_{11} e^{-i E_1 t} + c_2 O_{12} e^{-i E_2 t} + c_3 O_{13} e^{-i E_3 t} \right) + \mathellipsis
    \end{aligned}
\end{equation*}
From the above equation it is clear that without a reference state we will get energy differences and not energy eigenvalues.
We can fix this problem and measure the eigenvalues by including a reference state that is decoupled from the dynamics of the 3x3 subspace and serves as an interference partner for a Ramsey measurement.
The natural choice is the vacuum state.
Taking the reference (vacuum) state to have zero energy and decoupled from the manifold that we wish to measure the eigenvalues of:
\begin{equation*}
    \ket{\psi (t)} = c_0\ket{\phi_{ref}} + c_1\ket{\phi_1}e^{-i E_1 t} + c_2\ket{\phi_2}e^{-i E_2 t} + c_3\ket{\phi_3}e^{-i E_3 t}
\end{equation*}

and
\begin{equation*}
    O =
    \begin{bmatrix}
        O_{00} & O_{01} & O_{02} & O_{03}\\
        O_{10} & O_{11} & O_{12} & O_{13}\\
        O_{20} & O_{21} & O_{22} & O_{23}\\
        O_{30} & O_{31} & O_{32} & O_{33}\\
    \end{bmatrix}
\end{equation*}
which leads to


\begin{equation*}
    \begin{aligned}
        \braket{\psi(t)|\hat{O}|\psi(t)} = \\[2.0ex]
        \begin{bmatrix}
            c_0^* & c_1^* e^{i E_1 t} & c_2^* e^{i E_2 t} & c_3^* e^{i E_3 t}
        \end{bmatrix}
        \begin{bmatrix}
            O_{00} & O_{01} & O_{02} & O_{03}\\
            O_{10} & O_{11} & O_{12} & O_{13}\\
            O_{20} & O_{21} & O_{22} & O_{23}\\
            O_{30} & O_{31} & O_{32} & O_{33}\\
        \end{bmatrix}
        \begin{bmatrix}
            c_0  \\
            c_1 e^{-i E_1 t} \\
            c_2 e^{-i E_2 t} \\
            c_3 e^{-i E_3 t}
        \end{bmatrix} = \\[2.0 ex]
        c_0^* \left( c_0 O_{00} + c_1 O_{01} e^{-i E_1 t} + c_2 O_{02} e^{-i E_2 t} + c_3 O_{03} e^{-i E_3 t} \right)& + \\[2.0 ex]
        c_1^* e^{i E_1 t} \left( c_0 O_{10} + c_1 O_{11} e^{-i E_1 t} + c_2 O_{12} e^{-i E_2 t} + c_3 O_{13} e^{-i E_3 t} \right)& +\\[2.0 ex]
        c_2^* e^{i E_2 t} \left( c_0 O_{20} + c_1 O_{21} e^{-i E_1 t} + c_2 O_{22} e^{-i E_2 t} + c_3 O_{23} e^{-i E_3 t} \right)& +\\[2.0 ex]
        c_3^* e^{i E_3 t} \left( c_0 O_{30} + c_1 O_{31} e^{-i E_1 t} + c_2 O_{32} e^{-i E_2 t} + c_3 O_{33} e^{-i E_3 t} \right)& \\[2.0 ex]
    \end{aligned}
\end{equation*}


\subsection{simple example}
For analog control our measurement basis is different than the dynamic basis.
\begin{equation*}
    %\begin{aligned}
    \ket{\psi} \implies \textrm{the wavefunction}
    \ket{\varphi} \implies \textrm{The measurement / initialization / Fock basis}
    \ket{\phi} \implies \textrm{The measurement / Fock basis}
    %\end{aligned}
\end{equation*}

Consider the example of the vacuum + 1 photon subspace of a three qubit system:
the fock basis states are:
\begin{equation*}
    \varphi = \ket{000}, \ket{001}, \ket{010}, \ket{100}
\end{equation*}
Initializing to a superposition state by doing a X/2 pulse we have:
\begin{equation*}
    \psi = \frac{1}{\sqrt{2}} \left( \ket{000} + \ket{001} \right) = \frac{1}{\sqrt{2}} \left( \ket{\varphi_0} + \ket{\varphi_1} \right)
\end{equation*}

\begin{equation}
    H=
    \begin{bmatrix}
        1 & 0 & 0 & 0 \\
        0 & H_{11} & H_{12} & H_{13} \\
        0 & H_{21} & H_{22} & H_{23} \\
        0 & H_{31} & H_{32} & H_{33} \\
    \end{bmatrix}
\end{equation}


Skipping over some algebra that is in my notebook.

In the Energy basis
\begin{equation}
    \ket{\psi (t)} = \frac{1}{\sqrt{2}}
    \begin{bmatrix}
        1 \\
        a_{11} e^{-i E_1 t} + a_{12} e^{-i E_2 t} + a_{13} e^{-i E_3 t} \\
        a_{21} e^{-i E_1 t} + a_{22} e^{-i E_2 t} + a_{23} e^{-i E_3 t} \\
        a_{31} e^{-i E_1 t} + a_{32} e^{-i E_2 t} + a_{33} e^{-i E_3 t}
    \end{bmatrix}
\end{equation}

Now we can compare measurements with different operators on different initial states.
\begin{equation}
    \hat{a}_1 = \ket{000}\bra{001} =
    \begin{bmatrix}
        0 & 1 & 0 & 0 \\
        0 & 0 & 0 & 0 \\
        0 & 0 & 0 & 0 \\
        0 & 0 & 0 & 0 \\
    \end{bmatrix}
\end{equation}

\begin{equation}
    \braket{\psi(t)|\hat{a}_1|\psi(t)} = a_{11} e^{-i E_1 t} + a_{12} e^{-i E_2 t} + a_{13} e^{-i E_3 t}
\end{equation}

\begin{equation}
    \hat{n}_1 = \ket{001}\bra{001} =
    \begin{bmatrix}
        0 & 0 & 0 & 0 \\
        0 & 1 & 0 & 0 \\
        0 & 0 & 0 & 0 \\
        0 & 0 & 0 & 0 \\
    \end{bmatrix}
\end{equation}


\begin{equation}
    \hat{n}_1 = \ket{001}\bra{001} =
    \begin{bmatrix}
        0 & 0 & 0 & 0 \\
        0 & 1 & 0 & 0 \\
        0 & 0 & 0 & 0 \\
        0 & 0 & 0 & 0 \\
    \end{bmatrix}
\end{equation}

\begin{equation}
    \braket{\psi(t)|\hat{a}_1|\psi(t)} = a_{11} e^{-i E_1 t} + a_{12} e^{-i E_2 t} + a_{13} e^{-i E_3 t}
\end{equation}












\subsubsection{scratch}
\begin{equation}
    \hat{O}=\hat{a}_1 = \ket{000}\bra{001} \implies \textrm{O(t)} = \left( \mathellipsis \right)
\end{equation}

\begin{equation}
    \hat{O}=\hat{\sigma}^x_1 = \ket{000}\bra{001} + \ket{001}\bra{000} \implies \textrm{O(t)} = \left( \mathellipsis \right) + \left(  \mathellipsis  \right)^{*}
\end{equation}

\begin{equation}
    \hat{O}=\hat{n}_1 = \ket{001}\bra{001} \implies \textrm{O(t)} = \lvert \left( \mathellipsis \right) \rvert^{2}
\end{equation}









\section{Doublon Spectroscopy}
In the analog control project with the gmon we routinely work outside the qubit subspace.  It is necessary to find the eigenvalues of such states as $\ket{002}$ etc.
In this case we must engineer a similar operator with only one element.

Here we show how to engineer a measurement operator $\hat{O}$ including $\ket{2}$

From Sakurai The rotation operator for a spinor is

\begin{equation}
    \hat{D}(\hat{n},\phi)=e^{-i \hat{S} \cdot \hat{n} \frac{\phi}{\hbar}}=e^{-i \frac{\hat{\sigma} \cdot \hat{n}}{2} }
\end{equation}
which has the matrix representation:
\begin{equation}
    \begin{bmatrix}
        \cos \left( \frac{\phi}{2} \right) - i n_z \sin \left( {\frac{\phi}{2}} \right) & \left( -i n_x - n_y \right) \sin \left( \frac{\phi}{2} \right)   \\
        \left( -i n_x + n_y \right) \sin \left( \frac{\phi}{2} \right) & \cos \left( \frac{\phi}{2} \right) + i n_z \sin \left( {\frac{\phi}{2}} \right)   \\
    \end{bmatrix}
\end{equation}

For $\pi$ rotations this is:
\begin{equation}
    \hat{D}(\hat{X}, \pi)=
    \begin{bmatrix}
        0 & -i \\
        -i & 0
    \end{bmatrix}
\end{equation}
Including the $|2>$ and addressing only a single transition:

\begin{equation}
    \hat{D_{01}}(\hat{X}, \pi)=
    \begin{bmatrix}
        0  & -i & 0 \\
        -i & 0 & 0 \\
        0  & 0 & 1 \\
    \end{bmatrix},
    \hat{D_{01}}(\hat{-X}, \pi)=
    \begin{bmatrix}
        0  & i & 0 \\
        i & 0 & 0 \\
        0  & 0 & 1 \\
    \end{bmatrix},
\end{equation}
and
\begin{equation}
    \hat{D_{12}}(\hat{X}, \pi)=
    \begin{bmatrix}
        1 & 0 & 0 \\
        0 & 0 & -i \\
        0 & -i & 0 \\
    \end{bmatrix},
    \hat{D_{12}}(\hat{-X}, \pi)=
    \begin{bmatrix}
        1 & 0 & 0 \\
        0 & 0 & i \\
        0 & i & 0 \\
    \end{bmatrix}
\end{equation}

If we make an x rotation immediately before measuring x+iy our measurement operator is \textbf{TODO Check this} effectively:
\begin{equation}
    D_{12}^{\dagger}(x,\pi)(X_{01}+iY{01})D_{12}(X,\pi) =
    \begin{bmatrix}
        0 & 0 & -2 \\
        0 & 0 & 0 \\
        0 & (1+i) & 0 \\
    \end{bmatrix}
\end{equation}

\section{scratch}

\begin{equation}
    \hat{a}_1 = \sigma^{x} + i \sigma^{y} =
    \begin{bmatrix}
        0 & 1 \\
        1 & 0
    \end{bmatrix}
    +
    i
    \begin{bmatrix}
        0 & -i \\
        i & 0
    \end{bmatrix}
    =
    \begin{bmatrix}
        0 & 1 \\
        0 & 0
    \end{bmatrix}
\end{equation}

$\text{FFT}\left( O(t) \right)$

\begin{equation}
    \hat{a}
    \hat{\sigma}^{x}_{1}
\end{equation}

\begin{equation}
    \ket{\psi_{f}} = U \ket{\psi_{0}}
\end{equation}

\begin{equation}
    \frac{\ket{0} + \ket{1}}{\sqrt{2}}
\end{equation}


\begin{equation}
    \mathcal{H}=
    \begin{bmatrix}
        0 &   0&          0&          0 \\
        0 &   11.76+0.j&   17.14+7.65j& 12.7 +3.57j \\
        0 &   17.14-7.65j&  0.72+0.j&   12.1 +6.39j  \\
        0 &   12.7 -3.57j& 12.1 -6.39j& 13.16+0.j
    \end{bmatrix}
\end{equation}

\begin{equation}
    \ket{0}
    \ket{1}
    \ket{2}
    \ket{00}
    \ket{01}
    \ket{02}
    \ket{10} \\
    \ket{11}
    \ket{12}
    \ket{20}
    \ket{21}
    \ket{22} \\
    \ket{000}
    \ket{001}
    \ket{010}
    \ket{100}
    \ket{011}
    \ket{101} \\
    \ket{110}
    \ket{200}
    \ket{020}
    \ket{002}
\end{equation}

\begin{comment}
    \begin{equation}
        \bra{0}
        \bra{1}
        \bra{2} \\
        \bra{00}
        \bra{01}
        \bra{02}
        \bra{10}
        \bra{11}
        \bra{12}
        \bra{20}
        \bra{21}
        \bra{22}
        \bra{000} \\
        \bra{001}
        \bra{010}
        \bra{100}
        \bra{011}
        \bra{101}
        \bra{110}
        \bra{200}
        \bra{020}
        \bra{002} \\
        \Delta_{1}
        \Delta_{2}
        \Delta_{3}
        \Delta_{n-1}
        \Delta_{n} \\
        \textrm{g}_{1,2}
        \textrm{g}_{2,3}
        \textrm{g}_{(\textrm{n}-1),\textrm{n}} \\
        \textrm{XY}_{1} \\
        \sigma^{z}
        \textrm{q}_{\textrm{i}}
        \textrm{q}_{\textrm{m}}
        \frac{\textrm{Y}}{2}
        \textrm{U_{21}}
        \textrm{q1}
        \textrm{q2}
        \textrm{q3}
    \end{equation}
    \begin{equation}
        \ket{\psi_{0}} = \ket{001}
        \ket{\psi_{0}} = \frac{\ket{000} + \ket{001}}{\sqrt{2}}
        \textrm{O(t)}  = \lvert a e^{-i E_1 t} + b e^{-i E_2 t} + c e^{-i E_3 t} \rvert^{2}
        \textrm{O(t)} = \left( a e^{-i E_1 t} + b e^{-i E_2 t} + c e^{-i E_3 t}\right) + \left( a e^{-i E_1 t} + b e^{-i E_2 t} + c e^{-i E_3 t}\right)^{*}
        \textrm{O(t)} = \left( a e^{-i E_1 t} + b e^{-i E_2 t} + c e^{-i E_3 t}\right)
    \end{equation}
\end{comment}
\begin{comment}
    \begin{equation}
        -i
        \begin{bmatrix}
            \textrm{H}_{00} & \dots & \textrm{H}_{0\textrm{N}}\\
            \vdots & \ddots & \vdots \\
            \textrm{H}_{\textrm{N}0} & \dots & \textrm{H}_{\textrm{N}\textrm{N}}\\
        \end{bmatrix}
        t

        e^{
        \begin{bmatrix}
            \textrm{H}_{00} & \dots & \textrm{H}_{0\textrm{N}}\\
            \vdots & \ddots & \vdots \\
            \textrm{H}_{\textrm{N}0} & \dots & \textrm{H}_{\textrm{N}\textrm{N}}\\
        \end{bmatrix}
        }

    \end{equation}
\end{comment}

\begin{equation}
    H_{BH} = \sum_{i}^{N_{coupler}} g_{i,i+1} \left( a^\dagger_{i} a_{i+1} +a_{i} a^\dagger_{i+1} \right)+ \frac{U}{2}\sum_{i}^{N_{qubit}}n_i(n_i-1) + \sum_{i}^{N_{qubit}} h_i n_i
\end{equation}

\begin{equation}
    e^{-i \left( \hat{\text{I}} \hat{\text{Z}} - \hat{\text{Z}} \hat{\text{I}} \right) \frac{\delta_{a}}{4} }
    e^{-i \left( \hat{\text{X}} \hat{\text{X}} - \hat{\text{Y}} \hat{\text{Y}} \right) \frac{\theta}{2} }
    e^{-i \hat{\text{Z}} \hat{\text{Z}}  \frac{\phi}{4} }
    e^{-i \left( \hat{\text{I}} \hat{\text{Z}} - \hat{\text{Z}} \hat{\text{I}} \right) \frac{\delta_{b}}{4} }
    e^{-i \left( \hat{\text{I}} \hat{\text{Z}} + \hat{\text{Z}} \hat{\text{I}} \right) \frac{\omega}{4} }
\end{equation}

\begin{equation}
    U=
    \begin{bmatrix}
        1 & 0 & 0 & 0 \\
        0 & e^{ \frac{i}{2} (\omega)} \cos{\theta} & -i e^{ \frac{i}{2} (\omega)}\sin{\theta} & 0 \\
        0 & -i e^{ - \frac{i}{2} (- \omega)}\sin{\theta} & e^{ -\frac{i}{2} ( - \omega)}\cos{\theta} & 0 \\
        0 & 0 & 0 & e^{i (\theta-\phi)} \\
    \end{bmatrix}
\end{equation}


\begin{equation}
    \textrm{U off resonance}=
    \begin{bmatrix}
        1 & 0 & 0 & 0 \\
        0 & e^{ \frac{i}{2} (\delta_{a} + \delta_{b} + \omega)}\cos{\theta} & -i e^{ \frac{i}{2} (\delta_{a} - \delta_{b} + \omega)}\sin{\theta} & 0 \\
        0 & -i e^{ - \frac{i}{2} (\delta_{a} - \delta_{b} - \omega)}\sin{\theta} & e^{ -\frac{i}{2} (\delta_{a} + \delta_{b} - \omega)}\cos{\theta} & 0 \\
        0 & 0 & 0 & e^{i (\omega-\phi)} \\
    \end{bmatrix}
\end{equation}

\begin{equation}
    \text{U on resonance}=
    \begin{bmatrix}
        1 & 0 & 0 & 0 \\
        0 & e^{ i \frac{\theta}{2} } \cos{\theta} & -i e^{i \frac{\theta}{2} } \sin{\theta}  & 0 \\
        0 & -i  e^{ i \frac{\theta}{2} } \sin{\theta} & e^{i \frac{\theta}{2}} \cos{\theta} & 0 \\
        0 & 0 & 0 & e^{i (\theta-\phi)} \\
    \end{bmatrix}
\end{equation}

\begin{equation}
    \text{U on resonance simple}=
    \begin{bmatrix}
        1 & 0            &  0              & 0 \\
        0 & \cos{\theta} & -i \sin{\theta} & 0 \\
        0 & -i \sin{\theta} & \cos{\theta} & 0 \\
        0 &  0              & 0            & e^{i \phi} \\
    \end{bmatrix}
\end{equation}


\begin{equation}
    \delta_{a} = \delta_{b} = 0
\end{equation}

\begin{equation}
    \omega = \theta
\end{equation}

\begin{equation}
    \theta=0
\end{equation}

\begin{equation}
    \phi=0
\end{equation}

$\textrm{U}_{\textrm{fsim}}$
\begin{equation}
    U =
    \begin{bmatrix}
        1 & 0 & 0 & 0 \\
        0 & u_{11} & u_{12} & 0 \\
        0 & u_{21} & u_{22} & 0 \\
        0 & 0 & 0 & u_{33} \\
    \end{bmatrix}
\end{equation}

\begin{equation}
    \ket{\psi_0}=\frac{1}{\sqrt{2}}
    \begin{bmatrix}
        1 \\
        1 \\
        0 \\
        0 \\
    \end{bmatrix}
\end{equation}

\begin{equation}
    U\ket{\psi_0} =
    \frac{1}{\sqrt{2}}
    \begin{bmatrix}
        u00 + u01 \\
        u10 + u11 \\
        u20 + u21 \\
        u30 + u31 \\
    \end{bmatrix}
    =
    \frac{1}{\sqrt{2}}
    \begin{bmatrix}
        1 \\
        u_{11} \\
        u_{21} \\
        0 \\
    \end{bmatrix}
\end{equation}

\begin{equation}
    U=
    \begin{bmatrix}
        1 & 0 & 0  & 0 \\
        0 & u_{11} & u_{12} & 0 \\
        0 & u_{21} & u_{22} & 0 \\
        0 & 0 & 0  & u_{33} \\
    \end{bmatrix}
\end{equation}


\begin{equation}
    \hat{O}=\frac{1}{2}(\sigma^{x} + i\sigma^{y}) \otimes I=
    \begin{bmatrix}
        0 & 1\\
        0 & 0\\
    \end{bmatrix}
    \otimes
    \begin{bmatrix}
        1 & 0\\
        0 & 1\\
    \end{bmatrix}
    =
    \begin{bmatrix}
        0 & 0 & 1 & 0\\
        0 & 0 & 0 & 1\\
        0 & 0 & 0 & 0\\
        0 & 0 & 0 & 0\\
    \end{bmatrix}
\end{equation}
so that
\begin{equation}
    \ket{\psi_{f}}=
    \begin{bmatrix}
        1 \\
        u_{11} \\
        u_{21} \\
        0 \\
    \end{bmatrix}
    \implies
    \braket{\psi_{f}|\hat{O}|\psi_{f}} = u_{21}
\end{equation}
or if you prefer:
\begin{equation}
    \braket{\psi_{f}|\hat{O}|\psi_{f}} =
    \begin{bmatrix}
        1 & u_{11}^{*} & u_{21}^{*} & 0
    \end{bmatrix}
    \begin{bmatrix}
        0 & 0 & 1 & 0\\
        0 & 0 & 0 & 1\\
        0 & 0 & 0 & 0\\
        0 & 0 & 0 & 0\\
    \end{bmatrix}
    \begin{bmatrix}
        1 \\
        u_{11} \\
        u_{21} \\
        0 \\
    \end{bmatrix}
    = u_{21}
\end{equation}


\begin{equation}
U_{cz}
\begin{bmatrix}
1 &  &  & \\
& 1 &  & \\
&  & 1 & \\
&  &  & -1\\
\end{bmatrix}
\end{equation}




\begin{equation}
U_{\textrm{fSim}}
\begin{bmatrix}
1 &  &   & \\
& 1 &   & \\
&  & 1  & \\
&  &  & -1\\
\end{bmatrix}
\end{equation}

\begin{equation}
\left< \sigma^{x} \right>^{2} + \left< \sigma^{y} \right>^{2} + \left< \sigma^{z} \right>^{2}
\end{equation}


\begin{equation}
h_{i} \in \left[ -\textrm{W}, \textrm{W} \right] \textrm{MHz}
h_{i} \in \left[ 50, 50 \right] \textrm{MHz}
g_{i} \in \left[ 0,  45 \right] \textrm{MHz}
\end{equation}



\begin{equation}
\mathcal{H}_{BH} =\frac{\hbar}{2 \pi} \left(\vphantom{\sum\limits_{i}^{N_{q}}}\right.
\underbrace{\sum\limits_{i}^{N_{q} - 1} g_{i,i+1}(t) (a_i^\dagger a_{i+1} + a_i a_{i+1}^\dagger)}_{\text{Hopping interaction}} +
\underbrace{\frac{1}{2} \sum\limits_{i}^{N_q} \eta_i (n_i (n_i-1))}_{\text{Qubit nonlinearity}} +
\underbrace{\sum\limits_{i}^{N_q} \Delta_i (t) n_i + \frac {1}{2}}_{\text{Uncoupled oscillators}}
\left.\vphantom{\sum\limits_{i}^{N_{q}}}\right)
\label{BH_Hamiltonian}
\end{equation}

\begin{equation}
\hat{\sigma^{z}}=
\begin{bmatrix}
1 & 0\\
0 & -1\\
\end{bmatrix}
\end{equation}

\begin{comment}
\section{MBL scratch}

\begin{equation}
H_{\tau} = \sum_i \widetilde{h_i} \tau^z_i + \sum_{i,j} J_{ij} \tau^z_i \tau^z_j + \sum_{ijk} \tau^z_i \tau^z_j \tau^z_k + \mathellipsis = \sum_i \Delta_{i}(\left{ \tau^z_{j \neq i} \right}) \tau^z_i
\end{equation}

\begin{equation}
\Delta_i = \widetilde{h_i} + \sum_{j} J_{ij} \tau^z_j + \sum_{jk} \tau^z_j \tau^z_k + \mathellipsis
\end{equation}

\begin{equation}
\frac{\ket{0} + \ket{1}}{\sqrt{2}}
\end{equation}
\begin{equation}
\frac{\ket{0} + i \ket{1}}{\sqrt{2}}
\end{equation}

\begin{equation}
g \; t
\end{equation}

\begin{equation}
\left< \sigma_{\textrm{N} \right>_{\textrm{disorder}}
\end{equation}

\begin{equation}
\left< \textrm{N} \right>_{\textrm{disorder}}
\end{equation}

\begin{equation}
\left< \textrm{N} \right>_{\textrm{disorder}}
\end{equation}


\begin{equation}
\rho = \left< \sigma^{x} \right> \hat{x} + \left< \sigma^{y} \right> \hat{y} + \left< \sigma^{z} \right> \hat{z}
\end{equation}

\begin{equation}
\mid \rho \mid
\end{equation}

\begin{equation}
\textrm{tr} \left( \rho^{2} \right)
\end{equation}

\end{comment}
