\chapter[Outlook]{Outlook}

\section{Analog outlook MBL}

Outlook:

This thesis makes progress in two directions.  First it contributes to the developement of the quantum computing platform.  Second it contributes to the MBL physics.

Analog is useful because even in a digital gate based processor the underlying dynamics are analog.  Thus the parasitics are analog.  In fact you can consider the entanglement growth in the manybody localized phased to be a study of the parasitc entanglement growth of the fixed coupling xmon processor from not very many years ago.

Over the past twenty years superconducting qubits have made remarkable progress in no small part by leveraging concepts originally developed in the atomic physics community.  However modern, supremacy-era quantum processors are comprised of a lattice of interacting atoms, it stands to reason that we will benefit from adopting the mindset and techniques of condensed matter physics.  This thesis represents a conceptual step in that direction.

Also valuable because we need to bring a condensed matter mindset to the characterization of the  As we go to larger systems
From the perspective

One shortcoming of in our modeling capabilities is our reliance on an overconstrained circuit model.

We do not have a capability to tell which Hamiltonian as actually implemented on our hardware, only how far the eigenvalues were from the control model predictions.  In one branch of ongoing research we are pursuing time domain “Hamiltonan recovery” techniques.

A significant part of our ability to classify algorithm complexity results relies on ergodic dynamics.  However in the crossover region between MBL and fully
There has been suggestions that we can use entanglement entropy to bound the complexity of the algorithm.

\draftcomment{
\begin{itemize}
\item Complexity in the localized regime.
\item Tomography
\end{itemize}


\section{The promise of quantum computing}

\section{The role of analog in next generation quantum computing}
\begin{itemize}
\item We do not have a quantum computer yet, but today's devices can play a role in answering questions in condensed matter physics
\item All of the underlying dynamics are analog, characterize open systems, readout time trace, ...
\item 3 qubit unitary design
\item 2 qubit unitary design
\end{itemize}

\section{Challenges facing analog quantum simulators}

\begin{itemize}
\item Not universal
\item Not error correctable
\item Decoherence
\item Complex control
\end{itemize}

TODO list for analog simulators
\begin{itemize}
    \item Not universal
    \item Not error correctable
    \item Decoherence
\end{itemize}
}

