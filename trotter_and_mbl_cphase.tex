% arara: lualatex
\documentclass{article}
\usepackage[T1]{fontenc}
%\usepackage[latin9]{inputenc}
\usepackage[utf8]{inputenc}
\usepackage{amstext}
\usepackage{graphicx}
\usepackage{commath}

\usepackage{amssymb}
\usepackage{multirow}
\usepackage{amsmath,amssymb}
\usepackage{verbatim}
\usepackage{braket}
%\usepackage{float}
%\pagestyle{empty}
\usepackage{mathtools}
\DeclareRobustCommand{\qeqn}[1]{
\begin{equation}
    #1
\end{equation}
}

\begin{document}
\section{Comparison of analog dynamics and Trotter expansion on a gate based processor}
We define a few variables and summarize the note from Kostya regarding the gate count.
let
\qeqn{r \text{ be the number of Trotter steps}}
\qeqn{\epsilon \text{ the final error}}
\qeqn{n \text{ the number of qubits}}

We wish to answer the question:
\textit{``How many single and two qubit gates would be required to perform our analog evolution if decomposed into a Trotter expansion?"}

\subsection{XY model}

The simplest case is an XY model without local $Z$ fields
\qeqn{H = J \sum_i^n X_i X_{i+1} + Y_i Y_{i+1} }
We make several assumptions favorable to gates:
\begin{itemize}
\item Rather than using a ``Textbook" gate set, we take our two qubit gate to be a small angle Givens rotion.
\item We omit local $Z$ fields from the Hamiltonian.
\item We consider only $\ket{0}$ and $\ket{1}$ on each qubit.
\end{itemize}
Under these assumptions and assuming \textbf{perfect} gates Kostya computes:
\qeqn{\text{Total Trotter Error $\epsilon_{T}$ } =
\norm{
\left(
\prod_{k=1}^{\frac{n}{2}-1} e^{i \frac{t}{r} H_{2k,2k+1} }
\prod_{k=1}^{\frac{n}{2}-1} e^{i \frac{t}{r} H_{2k-1,2k} }
\right)^{r}
- e^{- i H t}
\leq
\frac{n t^2}{r}
}
}

In our experiment $J = 40 \text{MHz}$ so that $1/J = 25\text{ns}$.
Thus for 100 ns of evolution $t=4$ and $n=9$.
Therefore in order to achieve a Trotter error of 0.1 for 100 ns evolution we require a gate depth of
\qeqn{ r = \frac{9 (4)^2}{0.1} = 1440}

\subsection{XY model with decoherence}
If we assume error from decoherence is nonzero and equal per qubit per unit time in both the analog and gate case.
Kostya shows that:
\begin{itemize}
\item The optimal decomposition has decoherence error equal to Trotter error.
\item The error in the analog case is lower than the gate decomposition by at least a factor of 14.
\end{itemize}

\subsection{Including the $\ket{2}$ }
\begin{itemize}
    \item You need ancilla qubits to track the dynamics of the higher levels
    \item For a simple nearest neighbor system where the Ancilla are ideally placed we estimate the ``shuffeling" overhead to be a factor of 15
    \item Shuffeling overhead gets worse in full 2D systems
\end{itemize}






\section{Measuring nonlocal interactions with a conditional phase experiment}
\end{document}
